%% Generated by Sphinx.
\def\sphinxdocclass{report}
\documentclass[letterpaper,10pt,english]{sphinxmanual}
\ifdefined\pdfpxdimen
   \let\sphinxpxdimen\pdfpxdimen\else\newdimen\sphinxpxdimen
\fi \sphinxpxdimen=.75bp\relax

\PassOptionsToPackage{warn}{textcomp}
\usepackage[utf8]{inputenc}
\ifdefined\DeclareUnicodeCharacter
% support both utf8 and utf8x syntaxes
  \ifdefined\DeclareUnicodeCharacterAsOptional
    \def\sphinxDUC#1{\DeclareUnicodeCharacter{"#1}}
  \else
    \let\sphinxDUC\DeclareUnicodeCharacter
  \fi
  \sphinxDUC{00A0}{\nobreakspace}
  \sphinxDUC{2500}{\sphinxunichar{2500}}
  \sphinxDUC{2502}{\sphinxunichar{2502}}
  \sphinxDUC{2514}{\sphinxunichar{2514}}
  \sphinxDUC{251C}{\sphinxunichar{251C}}
  \sphinxDUC{2572}{\textbackslash}
\fi
\usepackage{cmap}
\usepackage[T1]{fontenc}
\usepackage{amsmath,amssymb,amstext}
\usepackage{babel}



\usepackage{times}
\expandafter\ifx\csname T@LGR\endcsname\relax
\else
% LGR was declared as font encoding
  \substitutefont{LGR}{\rmdefault}{cmr}
  \substitutefont{LGR}{\sfdefault}{cmss}
  \substitutefont{LGR}{\ttdefault}{cmtt}
\fi
\expandafter\ifx\csname T@X2\endcsname\relax
  \expandafter\ifx\csname T@T2A\endcsname\relax
  \else
  % T2A was declared as font encoding
    \substitutefont{T2A}{\rmdefault}{cmr}
    \substitutefont{T2A}{\sfdefault}{cmss}
    \substitutefont{T2A}{\ttdefault}{cmtt}
  \fi
\else
% X2 was declared as font encoding
  \substitutefont{X2}{\rmdefault}{cmr}
  \substitutefont{X2}{\sfdefault}{cmss}
  \substitutefont{X2}{\ttdefault}{cmtt}
\fi


\usepackage[Bjarne]{fncychap}
\usepackage[,numfigreset=1,mathnumfig]{sphinx}

\fvset{fontsize=\small}
\usepackage{geometry}


% Include hyperref last.
\usepackage{hyperref}
% Fix anchor placement for figures with captions.
\usepackage{hypcap}% it must be loaded after hyperref.
% Set up styles of URL: it should be placed after hyperref.
\urlstyle{same}

\addto\captionsenglish{\renewcommand{\contentsname}{Contents:}}

\usepackage{sphinxmessages}
\setcounter{tocdepth}{1}



\title{Linear Dynamical Systems}
\date{Aug 13, 2021}
\release{1.0.1}
\author{Kapa Kudaibergenov}
\newcommand{\sphinxlogo}{\vbox{}}
\renewcommand{\releasename}{Release}
\makeindex
\begin{document}

\pagestyle{empty}
\sphinxmaketitle
\pagestyle{plain}
\sphinxtableofcontents
\pagestyle{normal}
\phantomsection\label{\detokenize{index::doc}}



\chapter{LDS}
\label{\detokenize{modules:lds}}\label{\detokenize{modules::doc}}

\section{LDS package}
\label{\detokenize{LDS:lds-package}}\label{\detokenize{LDS::doc}}

\subsection{Subpackages}
\label{\detokenize{LDS:subpackages}}

\subsubsection{LDS.LDS package}
\label{\detokenize{LDS.LDS:lds-lds-package}}\label{\detokenize{LDS.LDS::doc}}

\paragraph{Subpackages}
\label{\detokenize{LDS.LDS:subpackages}}

\subparagraph{LDS.LDS.ds package}
\label{\detokenize{LDS.LDS.ds:lds-lds-ds-package}}\label{\detokenize{LDS.LDS.ds::doc}}

\subparagraph{Submodules}
\label{\detokenize{LDS.LDS.ds:submodules}}

\subparagraph{LDS.LDS.ds.dynamical\_system module}
\label{\detokenize{LDS.LDS.ds:module-LDS.LDS.ds.dynamical_system}}\label{\detokenize{LDS.LDS.ds:lds-lds-ds-dynamical-system-module}}\index{module@\spxentry{module}!LDS.LDS.ds.dynamical\_system@\spxentry{LDS.LDS.ds.dynamical\_system}}\index{LDS.LDS.ds.dynamical\_system@\spxentry{LDS.LDS.ds.dynamical\_system}!module@\spxentry{module}}
\sphinxAtStartPar
Originally the class comes from inputlds.py file.
\index{DynamicalSystem (class in LDS.LDS.ds.dynamical\_system)@\spxentry{DynamicalSystem}\spxextra{class in LDS.LDS.ds.dynamical\_system}}

\begin{fulllineitems}
\phantomsection\label{\detokenize{LDS.LDS.ds:LDS.LDS.ds.dynamical_system.DynamicalSystem}}\pysiglinewithargsret{\sphinxbfcode{\sphinxupquote{class }}\sphinxcode{\sphinxupquote{LDS.LDS.ds.dynamical\_system.}}\sphinxbfcode{\sphinxupquote{DynamicalSystem}}}{\emph{\DUrole{n}{matrix\_a}}, \emph{\DUrole{n}{matrix\_b}}, \emph{\DUrole{n}{matrix\_c}}, \emph{\DUrole{n}{matrix\_d}}, \emph{\DUrole{o}{**}\DUrole{n}{kwargs}}}{}
\sphinxAtStartPar
Bases: \sphinxcode{\sphinxupquote{object}}

\sphinxAtStartPar
Creates LDS.

\sphinxAtStartPar
Inits DynamicalSystem with four matrix args and
adds possibility of additional keywords in arguments.

\begin{DUlineblock}{0em}
\item[] G \sphinxhyphen{} matrix\_a
\item[] w \sphinxhyphen{} process noise
\item[] F\_dash \sphinxhyphen{} matrix\_c
\item[] v \sphinxhyphen{} sensor noise.
\end{DUlineblock}

\sphinxAtStartPar
If a matrix\_a is a number, transforms it into float
and makes d\sphinxhyphen{}state vector equal to 1.
If a matrix\_a is square y x y, set d equal to y.
If a matrix\_b is a number, transform it into float
and set n\sphinxhyphen{}input vector equal to 1.
matrix\_b can’t take place in case of single numbered
matrix\_a.
If matrix\_b is a matrix, number of its columns is assigned to n.
If matrix\_c is a number, transform it into float
and set m\sphinxhyphen{}observation vector equal to 1.
matrix\_c can be a number only if matrix\_a is a number too.
If matrix\_c is a matrix, number of its rows is assigned to m.
matrix\_d can’t be not zero number if matrix\_b is a matrix.
Number of columns of matrix\_d must be equal to n.
\begin{quote}\begin{description}
\item[{Parameters}] \leavevmode\begin{itemize}
\item {} 
\sphinxAtStartPar
\sphinxstyleliteralstrong{\sphinxupquote{matrix\_a}} \textendash{} Evolution, system, transfer or state matrix (G matrix).
Shape nxn.

\item {} 
\sphinxAtStartPar
\sphinxstyleliteralstrong{\sphinxupquote{matrix\_b}} \textendash{} Control matrix.

\item {} 
\sphinxAtStartPar
\sphinxstyleliteralstrong{\sphinxupquote{noise. Shape nx1. Shape of covariance matrix nxn.\%}} (\sphinxstyleliteralemphasis{\sphinxupquote{\%Processing}}) \textendash{} 

\item {} 
\sphinxAtStartPar
\sphinxstyleliteralstrong{\sphinxupquote{matrix\_c}} \textendash{} First derivative of the observation
direction(aka design matrix F(nxm)). Shape mxn.

\item {} 
\sphinxAtStartPar
\sphinxstyleliteralstrong{\sphinxupquote{matrix\_d}} \textendash{} Feedthrough matrix.

\item {} 
\sphinxAtStartPar
\sphinxstyleliteralstrong{\sphinxupquote{noise}}\sphinxstyleliteralstrong{\sphinxupquote{ or }}\sphinxstyleliteralstrong{\sphinxupquote{observational error. Shape mx1. Shape of covariance matrix mxm.\%}} (\sphinxstyleliteralemphasis{\sphinxupquote{\%Sensor}}) \textendash{} 

\end{itemize}

\item[{Optional arguments}] \leavevmode\begin{itemize}
\item {} 
\sphinxAtStartPar
\sphinxstylestrong{process\_noise} \textendash{} Processing noise w.

\item {} 
\sphinxAtStartPar
\sphinxstylestrong{observation\_noise} \textendash{} Observation noise v.

\item {} 
\sphinxAtStartPar
\sphinxstylestrong{timevarying\_multiplier\_b}

\item {} 
\sphinxAtStartPar
\sphinxstylestrong{corrupt\_probability}

\end{itemize}

\item[{Raises}] \leavevmode\begin{itemize}
\item {} 
\sphinxAtStartPar
\sphinxstyleliteralstrong{\sphinxupquote{KeyError}} \textendash{} in case of no additional keywords.

\item {} 
\sphinxAtStartPar
\sphinxstyleliteralstrong{\sphinxupquote{Exits in case of wrong format of a matrix.}} \textendash{} 

\item {} 
\sphinxAtStartPar
\sphinxstyleliteralstrong{\sphinxupquote{Exits in case of not square matrix\_a.}} \textendash{} 

\item {} 
\sphinxAtStartPar
\sphinxstyleliteralstrong{\sphinxupquote{Exits in case of having any matrix\_b}}\sphinxstyleliteralstrong{\sphinxupquote{, }}\sphinxstyleliteralstrong{\sphinxupquote{but matrix\_a is a number.}} \textendash{} 

\item {} 
\sphinxAtStartPar
\sphinxstyleliteralstrong{\sphinxupquote{Exits if number of rows of matrix\_b isn\textquotesingle{}t equal to d.}} \textendash{} 

\item {} 
\sphinxAtStartPar
\sphinxstyleliteralstrong{\sphinxupquote{Exits if matrix\_c is a number}}\sphinxstyleliteralstrong{\sphinxupquote{, }}\sphinxstyleliteralstrong{\sphinxupquote{but matrix\_a is not.}} \textendash{} 

\item {} 
\sphinxAtStartPar
\sphinxstyleliteralstrong{\sphinxupquote{Exits if number of columns of matrix\_c is not equal to d.}} \textendash{} 

\item {} 
\sphinxAtStartPar
\sphinxstyleliteralstrong{\sphinxupquote{Exits if matrix\_b is a matrix}}\sphinxstyleliteralstrong{\sphinxupquote{, }}\sphinxstyleliteralstrong{\sphinxupquote{but matrix\_d is not zero number.}} \textendash{} 

\item {} 
\sphinxAtStartPar
\sphinxstyleliteralstrong{\sphinxupquote{Exits if number of columns of matrix\_d is not equal to n\sphinxhyphen{}input vector.}} \textendash{} 

\end{itemize}

\end{description}\end{quote}
\index{check\_input() (LDS.LDS.ds.dynamical\_system.DynamicalSystem method)@\spxentry{check\_input()}\spxextra{LDS.LDS.ds.dynamical\_system.DynamicalSystem method}}

\begin{fulllineitems}
\phantomsection\label{\detokenize{LDS.LDS.ds:LDS.LDS.ds.dynamical_system.DynamicalSystem.check_input}}\pysiglinewithargsret{\sphinxbfcode{\sphinxupquote{check\_input}}}{\emph{\DUrole{n}{operator}}}{}
\sphinxAtStartPar
Checks variable type of matrices A,B,C,D.
\begin{quote}\begin{description}
\item[{Parameters}] \leavevmode
\sphinxAtStartPar
\sphinxstyleliteralstrong{\sphinxupquote{operator}} \textendash{} Number or a matrix.

\item[{Returns}] \leavevmode
\sphinxAtStartPar
1

\item[{Raises}] \leavevmode
\sphinxAtStartPar
\sphinxstyleliteralstrong{\sphinxupquote{TypeError}} \textendash{} This error occurs if the argument is none of possible formats.

\end{description}\end{quote}

\end{fulllineitems}

\index{solve() (LDS.LDS.ds.dynamical\_system.DynamicalSystem method)@\spxentry{solve()}\spxextra{LDS.LDS.ds.dynamical\_system.DynamicalSystem method}}

\begin{fulllineitems}
\phantomsection\label{\detokenize{LDS.LDS.ds:LDS.LDS.ds.dynamical_system.DynamicalSystem.solve}}\pysiglinewithargsret{\sphinxbfcode{\sphinxupquote{solve}}}{\emph{\DUrole{n}{h\_zero}}, \emph{\DUrole{n}{inputs}}, \emph{\DUrole{n}{t\_t}}, \emph{\DUrole{o}{**}\DUrole{n}{kwargs}}}{}
\sphinxAtStartPar
Finds outputs of LDS. The function is used in filters to find the error of prediction.

\sphinxAtStartPar
t\_t must be an integer greater than 1.
Length of h\_zero array must be equal to
self.d(number of arrays in matrix A) if matrix\_a
is matrix
If self.n\sphinxhyphen{}input vector is 1(matrix\_b is a number),
self.inputs will be transformed to a columns with t\_t
size.
If matrix\_b is matrix, inputs must have n x t\_t size.
If self.process\_noise has Gaussian distribution, we
create it with size d x t\_t. If it isn’t of Gaussian,
we create matrix of zeros.
If self.observation\_noise has Gaussian distribution, we
create it with size m x t\_t. If it isn’t of Gaussian,
we create matrix of zeros.
If it’s wasn’t given in init, we put earlies\_event\_time
to zero.
\begin{quote}\begin{description}
\item[{Parameters}] \leavevmode\begin{itemize}
\item {} 
\sphinxAtStartPar
\sphinxstyleliteralstrong{\sphinxupquote{h\_zero}} \textendash{} 1x2 array.

\item {} 
\sphinxAtStartPar
\sphinxstyleliteralstrong{\sphinxupquote{inputs}} \textendash{} Array of zeros of t\_t size.

\item {} 
\sphinxAtStartPar
\sphinxstyleliteralstrong{\sphinxupquote{t\_t}} \textendash{} Time horizon.

\end{itemize}

\item[{Optional arguments}] \leavevmode
\sphinxAtStartPar
\sphinxstylestrong{earliest\_event\_time}

\item[{Raises}] \leavevmode\begin{itemize}
\item {} 
\sphinxAtStartPar
\sphinxstyleliteralstrong{\sphinxupquote{Exits if t\_t is 1}}\sphinxstyleliteralstrong{\sphinxupquote{ or }}\sphinxstyleliteralstrong{\sphinxupquote{a float.}} \textendash{} 

\item {} 
\sphinxAtStartPar
\sphinxstyleliteralstrong{\sphinxupquote{Exits if matrix\_a is a number}}\sphinxstyleliteralstrong{\sphinxupquote{, }}\sphinxstyleliteralstrong{\sphinxupquote{but h\_zero can\textquotesingle{}t be transformed into float.}} \textendash{} 

\item {} 
\sphinxAtStartPar
\sphinxstyleliteralstrong{\sphinxupquote{Exits if length of h\_zero isn\textquotesingle{}t equal to d}}\sphinxstyleliteralstrong{\sphinxupquote{(}}\sphinxstyleliteralstrong{\sphinxupquote{if matrix\_a is matrix}}\sphinxstyleliteralstrong{\sphinxupquote{)}}\sphinxstyleliteralstrong{\sphinxupquote{}} \textendash{} 

\item {} 
\sphinxAtStartPar
\sphinxstyleliteralstrong{\sphinxupquote{Exits if self.n==1}}\sphinxstyleliteralstrong{\sphinxupquote{, }}\sphinxstyleliteralstrong{\sphinxupquote{but inputs don\textquotesingle{}t have a size of t\_t.}} \textendash{} 

\item {} 
\sphinxAtStartPar
\sphinxstyleliteralstrong{\sphinxupquote{Exits if matrix\_b is a matrix}}\sphinxstyleliteralstrong{\sphinxupquote{, }}\sphinxstyleliteralstrong{\sphinxupquote{but inputs don\textquotesingle{}t have n x t\_t size.}} \textendash{} 

\end{itemize}

\end{description}\end{quote}

\end{fulllineitems}


\end{fulllineitems}



\subparagraph{Module contents}
\label{\detokenize{LDS.LDS.ds:module-LDS.LDS.ds}}\label{\detokenize{LDS.LDS.ds:module-contents}}\index{module@\spxentry{module}!LDS.LDS.ds@\spxentry{LDS.LDS.ds}}\index{LDS.LDS.ds@\spxentry{LDS.LDS.ds}!module@\spxentry{module}}

\subparagraph{LDS.LDS.filters package}
\label{\detokenize{LDS.LDS.filters:lds-lds-filters-package}}\label{\detokenize{LDS.LDS.filters::doc}}

\subparagraph{Submodules}
\label{\detokenize{LDS.LDS.filters:submodules}}

\subparagraph{LDS.LDS.filters.filtering\_abc\_class module}
\label{\detokenize{LDS.LDS.filters:module-LDS.LDS.filters.filtering_abc_class}}\label{\detokenize{LDS.LDS.filters:lds-lds-filters-filtering-abc-class-module}}\index{module@\spxentry{module}!LDS.LDS.filters.filtering\_abc\_class@\spxentry{LDS.LDS.filters.filtering\_abc\_class}}\index{LDS.LDS.filters.filtering\_abc\_class@\spxentry{LDS.LDS.filters.filtering\_abc\_class}!module@\spxentry{module}}
\sphinxAtStartPar
This script implements ABC class.
\index{Filtering (class in LDS.LDS.filters.filtering\_abc\_class)@\spxentry{Filtering}\spxextra{class in LDS.LDS.filters.filtering\_abc\_class}}

\begin{fulllineitems}
\phantomsection\label{\detokenize{LDS.LDS.filters:LDS.LDS.filters.filtering_abc_class.Filtering}}\pysiglinewithargsret{\sphinxbfcode{\sphinxupquote{class }}\sphinxcode{\sphinxupquote{LDS.LDS.filters.filtering\_abc\_class.}}\sphinxbfcode{\sphinxupquote{Filtering}}}{\emph{\DUrole{n}{sys}}, \emph{\DUrole{n}{t\_t}}}{}
\sphinxAtStartPar
Bases: \sphinxcode{\sphinxupquote{abc.ABC}}

\sphinxAtStartPar
Abstract class for creation of filters.

\sphinxAtStartPar
Hierarchy tree ((ABC)):

\noindent\sphinxincludegraphics{{asciiart-0aa80faa7d7ebaca351fdd8bff7b40737ff7e143}.png}

\sphinxAtStartPar
Initializing a basic filter.
\begin{quote}\begin{description}
\item[{Parameters}] \leavevmode\begin{itemize}
\item {} 
\sphinxAtStartPar
\sphinxstyleliteralstrong{\sphinxupquote{sys}} \textendash{} LDS. DynamicalSystem object.

\item {} 
\sphinxAtStartPar
\sphinxstyleliteralstrong{\sphinxupquote{t\_t}} \textendash{} Time horizon.

\end{itemize}

\end{description}\end{quote}

\end{fulllineitems}



\subparagraph{LDS.LDS.filters.filtering\_siso module}
\label{\detokenize{LDS.LDS.filters:module-LDS.LDS.filters.filtering_siso}}\label{\detokenize{LDS.LDS.filters:lds-lds-filters-filtering-siso-module}}\index{module@\spxentry{module}!LDS.LDS.filters.filtering\_siso@\spxentry{LDS.LDS.filters.filtering\_siso}}\index{LDS.LDS.filters.filtering\_siso@\spxentry{LDS.LDS.filters.filtering\_siso}!module@\spxentry{module}}
\sphinxAtStartPar
This script implements ABC class.
\index{FilteringSiso (class in LDS.LDS.filters.filtering\_siso)@\spxentry{FilteringSiso}\spxextra{class in LDS.LDS.filters.filtering\_siso}}

\begin{fulllineitems}
\phantomsection\label{\detokenize{LDS.LDS.filters:LDS.LDS.filters.filtering_siso.FilteringSiso}}\pysiglinewithargsret{\sphinxbfcode{\sphinxupquote{class }}\sphinxcode{\sphinxupquote{LDS.LDS.filters.filtering\_siso.}}\sphinxbfcode{\sphinxupquote{FilteringSiso}}}{\emph{\DUrole{n}{sys}}, \emph{\DUrole{n}{t\_t}}}{}
\sphinxAtStartPar
Bases: {\hyperref[\detokenize{LDS.LDS.filters:LDS.LDS.filters.filtering_abc_class.Filtering}]{\sphinxcrossref{\sphinxcode{\sphinxupquote{LDS.LDS.filters.filtering\_abc\_class.Filtering}}}}}

\sphinxAtStartPar
Abstract class.
Specifically written to separate Kalman filter and auto\sphinxhyphen{}regression from spectral and
persistent filters.

\sphinxAtStartPar
Hierarchy tree ((ABC)):

\noindent\sphinxincludegraphics{{asciiart-2b351840c3bc463ea62ff24bbf7ff04fb7d27ac3}.png}

\sphinxAtStartPar
Inherits init method of Filtering.
\begin{quote}\begin{description}
\item[{Parameters}] \leavevmode\begin{itemize}
\item {} 
\sphinxAtStartPar
\sphinxstyleliteralstrong{\sphinxupquote{sys}} \textendash{} LDS. DynamicalSystem object.

\item {} 
\sphinxAtStartPar
\sphinxstyleliteralstrong{\sphinxupquote{t\_t}} \textendash{} Time horizon.

\end{itemize}

\end{description}\end{quote}
\index{predict() (LDS.LDS.filters.filtering\_siso.FilteringSiso method)@\spxentry{predict()}\spxextra{LDS.LDS.filters.filtering\_siso.FilteringSiso method}}

\begin{fulllineitems}
\phantomsection\label{\detokenize{LDS.LDS.filters:LDS.LDS.filters.filtering_siso.FilteringSiso.predict}}\pysiglinewithargsret{\sphinxbfcode{\sphinxupquote{abstract }}\sphinxbfcode{\sphinxupquote{predict}}}{}{}
\sphinxAtStartPar
Creates empty lists for prediction and error of filters.
\begin{quote}\begin{description}
\item[{Returns}] \leavevmode
\sphinxAtStartPar

\sphinxAtStartPar
tuple containing:
\begin{itemize}
\item {} 
\sphinxAtStartPar
y\_pred\_full : Output prediction.

\item {} 
\sphinxAtStartPar
pred\_error  : Prediction error.

\end{itemize}


\item[{Return type}] \leavevmode
\sphinxAtStartPar
(tuple)

\end{description}\end{quote}

\end{fulllineitems}


\end{fulllineitems}



\subparagraph{LDS.LDS.filters.kalman\_em module}
\label{\detokenize{LDS.LDS.filters:module-LDS.LDS.filters.kalman_em}}\label{\detokenize{LDS.LDS.filters:lds-lds-filters-kalman-em-module}}\index{module@\spxentry{module}!LDS.LDS.filters.kalman\_em@\spxentry{LDS.LDS.filters.kalman\_em}}\index{LDS.LDS.filters.kalman\_em@\spxentry{LDS.LDS.filters.kalman\_em}!module@\spxentry{module}}

\subparagraph{LDS.LDS.filters.kalman\_filtering\_siso module}
\label{\detokenize{LDS.LDS.filters:module-LDS.LDS.filters.kalman_filtering_siso}}\label{\detokenize{LDS.LDS.filters:lds-lds-filters-kalman-filtering-siso-module}}\index{module@\spxentry{module}!LDS.LDS.filters.kalman\_filtering\_siso@\spxentry{LDS.LDS.filters.kalman\_filtering\_siso}}\index{LDS.LDS.filters.kalman\_filtering\_siso@\spxentry{LDS.LDS.filters.kalman\_filtering\_siso}!module@\spxentry{module}}
\sphinxAtStartPar
Implements Kalman filter prediction.
Originates from function Kalman\_filtering\_SISO from onlinelds.py.
\index{KalmanFilteringSISO (class in LDS.LDS.filters.kalman\_filtering\_siso)@\spxentry{KalmanFilteringSISO}\spxextra{class in LDS.LDS.filters.kalman\_filtering\_siso}}

\begin{fulllineitems}
\phantomsection\label{\detokenize{LDS.LDS.filters:LDS.LDS.filters.kalman_filtering_siso.KalmanFilteringSISO}}\pysiglinewithargsret{\sphinxbfcode{\sphinxupquote{class }}\sphinxcode{\sphinxupquote{LDS.LDS.filters.kalman\_filtering\_siso.}}\sphinxbfcode{\sphinxupquote{KalmanFilteringSISO}}}{\emph{\DUrole{n}{sys}}, \emph{\DUrole{n}{G}}, \emph{\DUrole{n}{f\_dash}}, \emph{\DUrole{n}{proc\_noise\_std}}, \emph{\DUrole{n}{obs\_noise\_std}}, \emph{\DUrole{n}{t\_t}}, \emph{\DUrole{n}{Y}}}{}
\sphinxAtStartPar
Bases: {\hyperref[\detokenize{LDS.LDS.filters:LDS.LDS.filters.filtering_siso.FilteringSiso}]{\sphinxcrossref{\sphinxcode{\sphinxupquote{LDS.LDS.filters.filtering\_siso.FilteringSiso}}}}}

\sphinxAtStartPar
Calculates Kalman filter parameters. Finds the prediction for Kalman and auto\sphinxhyphen{}regression.
Uses abstract superclass FilteringSiso.

\sphinxAtStartPar
\hyperref[\detokenize{LDS.LDS.filters:kalman}]{Fig.\@ \ref{\detokenize{LDS.LDS.filters:kalman}}} shows an example of predictions of the Kalman, spectral and persistent filters.
\(G\) and \(F'\) are random normally distributed matrices.

\sphinxAtStartPar
Hierarchy tree ((ABC)):

\noindent\sphinxincludegraphics{{asciiart-6b8b88f392d97e48f09e08958d1be81eaa622e44}.png}

\begin{figure}[htbp]
\centering
\capstart

\noindent\sphinxincludegraphics{{Kalman}.png}
\caption{The prediction of the Kalman filter, persistent filter and spectral filter compared to the
real outputs over N = 100 iterations.}\label{\detokenize{LDS.LDS.filters:id1}}\label{\detokenize{LDS.LDS.filters:kalman}}\end{figure}

\sphinxAtStartPar
Inherits init method of FilteringSiso. Assignment variable names to LDS matrices.
Calls to method “parameters”.
\begin{quote}\begin{description}
\item[{Parameters}] \leavevmode\begin{itemize}
\item {} 
\sphinxAtStartPar
\sphinxstyleliteralstrong{\sphinxupquote{sys}} \textendash{} LDS. DynamicalSystem object.

\item {} 
\sphinxAtStartPar
\sphinxstyleliteralstrong{\sphinxupquote{G}} \textendash{} State transition matrix. Shape nxn.

\item {} 
\sphinxAtStartPar
\sphinxstyleliteralstrong{\sphinxupquote{f\_dash}} \textendash{} Observation direction. Shape mxn.

\item {} 
\sphinxAtStartPar
\sphinxstyleliteralstrong{\sphinxupquote{proc\_noise\_std}} \textendash{} Standard deviation of processing noise.

\item {} 
\sphinxAtStartPar
\sphinxstyleliteralstrong{\sphinxupquote{obs\_noise\_std}} \textendash{} Standard deviation of observation noise.

\item {} 
\sphinxAtStartPar
\sphinxstyleliteralstrong{\sphinxupquote{t\_t}} \textendash{} Time horizon.

\item {} 
\sphinxAtStartPar
\sphinxstyleliteralstrong{\sphinxupquote{Y}} \textendash{} Observations.

\end{itemize}

\end{description}\end{quote}
\index{parameters() (LDS.LDS.filters.kalman\_filtering\_siso.KalmanFilteringSISO method)@\spxentry{parameters()}\spxextra{LDS.LDS.filters.kalman\_filtering\_siso.KalmanFilteringSISO method}}

\begin{fulllineitems}
\phantomsection\label{\detokenize{LDS.LDS.filters:LDS.LDS.filters.kalman_filtering_siso.KalmanFilteringSISO.parameters}}\pysiglinewithargsret{\sphinxbfcode{\sphinxupquote{parameters}}}{}{}
\sphinxAtStartPar
Finds Kalman filter’s parameters:
\begin{quote}\begin{description}
\item[{Parameters}] \leavevmode\begin{itemize}
\item {} 
\sphinxAtStartPar
\sphinxstyleliteralstrong{\sphinxupquote{n}} \textendash{} Input vector. Shape of processing noise.

\item {} 
\sphinxAtStartPar
\sphinxstyleliteralstrong{\sphinxupquote{m}} \textendash{} Observation vector. Shape of observational error.

\item {} 
\sphinxAtStartPar
\sphinxstyleliteralstrong{\sphinxupquote{W}} \textendash{} Processing noise covariance.

\item {} 
\sphinxAtStartPar
\sphinxstyleliteralstrong{\sphinxupquote{V}} \textendash{} Observation noise covariance.

\item {} 
\sphinxAtStartPar
\sphinxstyleliteralstrong{\sphinxupquote{matrix\_c}} \textendash{} Covariance matrix of state.

\item {} 
\sphinxAtStartPar
\sphinxstyleliteralstrong{\sphinxupquote{R}} \textendash{} Covariance matrix of observation noise.

\item {} 
\sphinxAtStartPar
\sphinxstyleliteralstrong{\sphinxupquote{Q}} \textendash{} Covariance matrix of processing noise.

\item {} 
\sphinxAtStartPar
\sphinxstyleliteralstrong{\sphinxupquote{matrix\_a}} \textendash{} Kalman filter parameter. Not the same as matrix\_a in DynamicalSystem class.

\item {} 
\sphinxAtStartPar
\sphinxstyleliteralstrong{\sphinxupquote{Z}} \textendash{} Kalman filter parameter.

\end{itemize}

\end{description}\end{quote}
\begin{description}
\item[{Raises:  \#Not raises yet}] \leavevmode
\sphinxAtStartPar
Q can’t be zero.

\end{description}

\end{fulllineitems}

\index{predict() (LDS.LDS.filters.kalman\_filtering\_siso.KalmanFilteringSISO method)@\spxentry{predict()}\spxextra{LDS.LDS.filters.kalman\_filtering\_siso.KalmanFilteringSISO method}}

\begin{fulllineitems}
\phantomsection\label{\detokenize{LDS.LDS.filters:LDS.LDS.filters.kalman_filtering_siso.KalmanFilteringSISO.predict}}\pysiglinewithargsret{\sphinxbfcode{\sphinxupquote{predict}}}{\emph{\DUrole{n}{s}}, \emph{\DUrole{n}{error\_AR1\_data}}, \emph{\DUrole{n}{error\_kalman\_data}}}{}
\sphinxAtStartPar
Calculates output predictions and errors for auto\sphinxhyphen{}regression and Kalman filter.
\begin{quote}\begin{description}
\item[{Parameters}] \leavevmode\begin{itemize}
\item {} 
\sphinxAtStartPar
\sphinxstyleliteralstrong{\sphinxupquote{s}} \textendash{} Auto\sphinxhyphen{}regression depth.

\item {} 
\sphinxAtStartPar
\sphinxstyleliteralstrong{\sphinxupquote{error\_AR1\_data}} \textendash{} Auto\sphinxhyphen{}regression error. 2\sphinxhyphen{}norm.

\item {} 
\sphinxAtStartPar
\sphinxstyleliteralstrong{\sphinxupquote{error\_kalman\_data}} \textendash{} Kalman error. 2\sphinxhyphen{}norm.

\end{itemize}

\item[{Returns}] \leavevmode
\sphinxAtStartPar

\sphinxAtStartPar
tuple containing:
\begin{itemize}
\item {} 
\sphinxAtStartPar
y\_pred\_full       : Output prediction.

\item {} 
\sphinxAtStartPar
error\_AR1\_data    : Auto\sphinxhyphen{}regression error. 2\sphinxhyphen{}norm.

\item {} 
\sphinxAtStartPar
error\_kalman\_data : Kalman error. 2\sphinxhyphen{}norm.

\end{itemize}


\item[{Return type}] \leavevmode
\sphinxAtStartPar
(tuple)

\end{description}\end{quote}

\end{fulllineitems}

\index{predict\_kalman() (LDS.LDS.filters.kalman\_filtering\_siso.KalmanFilteringSISO method)@\spxentry{predict\_kalman()}\spxextra{LDS.LDS.filters.kalman\_filtering\_siso.KalmanFilteringSISO method}}

\begin{fulllineitems}
\phantomsection\label{\detokenize{LDS.LDS.filters:LDS.LDS.filters.kalman_filtering_siso.KalmanFilteringSISO.predict_kalman}}\pysiglinewithargsret{\sphinxbfcode{\sphinxupquote{predict\_kalman}}}{\emph{\DUrole{n}{s}}, \emph{\DUrole{n}{error\_AR1\_data}}, \emph{\DUrole{n}{error\_kalman\_data}}}{}
\sphinxAtStartPar
Calculates output predictions and errors for auto\sphinxhyphen{}regression and Kalman filter.
\begin{quote}\begin{description}
\item[{Parameters}] \leavevmode\begin{itemize}
\item {} 
\sphinxAtStartPar
\sphinxstyleliteralstrong{\sphinxupquote{s}} \textendash{} Auto\sphinxhyphen{}regression depth.

\item {} 
\sphinxAtStartPar
\sphinxstyleliteralstrong{\sphinxupquote{error\_AR1\_data}} \textendash{} Auto\sphinxhyphen{}regression error. 2\sphinxhyphen{}norm.

\item {} 
\sphinxAtStartPar
\sphinxstyleliteralstrong{\sphinxupquote{error\_kalman\_data}} \textendash{} Kalman error. 2\sphinxhyphen{}norm.

\end{itemize}

\item[{Returns}] \leavevmode
\sphinxAtStartPar

\sphinxAtStartPar
tuple containing:
\begin{itemize}
\item {} 
\sphinxAtStartPar
y\_pred\_full       : Output prediction.

\item {} 
\sphinxAtStartPar
error\_AR1\_data    : Auto\sphinxhyphen{}regression error. 2\sphinxhyphen{}norm.

\item {} 
\sphinxAtStartPar
error\_kalman\_data : Kalman error. 2\sphinxhyphen{}norm.

\end{itemize}


\item[{Return type}] \leavevmode
\sphinxAtStartPar
(tuple)

\end{description}\end{quote}

\end{fulllineitems}


\end{fulllineitems}



\subparagraph{LDS.LDS.filters.wave\_filtering\_siso module}
\label{\detokenize{LDS.LDS.filters:module-LDS.LDS.filters.wave_filtering_siso}}\label{\detokenize{LDS.LDS.filters:lds-lds-filters-wave-filtering-siso-module}}\index{module@\spxentry{module}!LDS.LDS.filters.wave\_filtering\_siso@\spxentry{LDS.LDS.filters.wave\_filtering\_siso}}\index{LDS.LDS.filters.wave\_filtering\_siso@\spxentry{LDS.LDS.filters.wave\_filtering\_siso}!module@\spxentry{module}}
\sphinxAtStartPar
Originates from function wave\_filtering\_SISO from onlineLDS.py.
The related work is “Learning Linear Dynamical Systems via Spectral Filtering”
by E.Hazan, K.Singh and C.Zhang.
\index{WaveFilteringSISO (class in LDS.LDS.filters.wave\_filtering\_siso)@\spxentry{WaveFilteringSISO}\spxextra{class in LDS.LDS.filters.wave\_filtering\_siso}}

\begin{fulllineitems}
\phantomsection\label{\detokenize{LDS.LDS.filters:LDS.LDS.filters.wave_filtering_siso.WaveFilteringSISO}}\pysiglinewithargsret{\sphinxbfcode{\sphinxupquote{class }}\sphinxcode{\sphinxupquote{LDS.LDS.filters.wave\_filtering\_siso.}}\sphinxbfcode{\sphinxupquote{WaveFilteringSISO}}}{\emph{\DUrole{n}{sys}}, \emph{\DUrole{n}{t\_t}}, \emph{\DUrole{n}{k}}, \emph{\DUrole{n}{eta}}, \emph{\DUrole{n}{r\_m}}}{}
\sphinxAtStartPar
Bases: {\hyperref[\detokenize{LDS.LDS.filters:LDS.LDS.filters.wave_filtering_siso_abs.WaveFilteringSisoAbs}]{\sphinxcrossref{\sphinxcode{\sphinxupquote{LDS.LDS.filters.wave\_filtering\_siso\_abs.WaveFilteringSisoAbs}}}}}

\sphinxAtStartPar
Implements spectral filter. \hyperref[\detokenize{LDS.LDS.filters:spectral}]{Fig.\@ \ref{\detokenize{LDS.LDS.filters:spectral}}} shows an example of its real\sphinxhyphen{}time prediction.

\sphinxAtStartPar
Hierarchy tree ((ABC)):

\noindent\sphinxincludegraphics{{asciiart-efc0743aaf27ac7bd5adf5e97451abdb799f0c96}.png}

\begin{figure}[htbp]
\centering
\capstart

\noindent\sphinxincludegraphics{{spectral}.png}
\caption{The prediction of the spectral filter compared to the real outputs over N = 100 iterations.}\label{\detokenize{LDS.LDS.filters:id2}}\label{\detokenize{LDS.LDS.filters:spectral}}\end{figure}

\sphinxAtStartPar
Inits all the attributes of its superclass(see WaveFilteringSisoAbs) and
adds Learning rate and Radius parameter.
Goes through all the methods and gets the predictions.
\begin{quote}\begin{description}
\item[{Parameters}] \leavevmode\begin{itemize}
\item {} 
\sphinxAtStartPar
\sphinxstyleliteralstrong{\sphinxupquote{sys}} \textendash{} LDS. DynamicalSystem object.

\item {} 
\sphinxAtStartPar
\sphinxstyleliteralstrong{\sphinxupquote{t\_t}} \textendash{} Time horizon.

\item {} 
\sphinxAtStartPar
\sphinxstyleliteralstrong{\sphinxupquote{k}} \textendash{} Number of wave\sphinxhyphen{}filters for a spectral filter.

\item {} 
\sphinxAtStartPar
\sphinxstyleliteralstrong{\sphinxupquote{eta}} \textendash{} Learning rate.

\item {} 
\sphinxAtStartPar
\sphinxstyleliteralstrong{\sphinxupquote{r\_m}} \textendash{} Radius parameter.

\end{itemize}

\end{description}\end{quote}
\index{predict() (LDS.LDS.filters.wave\_filtering\_siso.WaveFilteringSISO method)@\spxentry{predict()}\spxextra{LDS.LDS.filters.wave\_filtering\_siso.WaveFilteringSISO method}}

\begin{fulllineitems}
\phantomsection\label{\detokenize{LDS.LDS.filters:LDS.LDS.filters.wave_filtering_siso.WaveFilteringSISO.predict}}\pysiglinewithargsret{\sphinxbfcode{\sphinxupquote{predict}}}{}{}
\sphinxAtStartPar
Calculation of output predictions and prediction errors.
\begin{quote}\begin{description}
\item[{Returns}] \leavevmode
\sphinxAtStartPar

\sphinxAtStartPar
tuple containing:
\begin{itemize}
\item {} 
\sphinxAtStartPar
y\_pred\_full : Prediction values.

\item {} \begin{description}
\item[{M}] \leavevmode{[}Matrix specifying a linear map{]}
\sphinxAtStartPar
from featurized inputs to predictions.

\end{description}

\item {} 
\sphinxAtStartPar
pred\_error  : Spectral filter error.

\end{itemize}


\item[{Return type}] \leavevmode
\sphinxAtStartPar
(tuple)

\end{description}\end{quote}

\end{fulllineitems}


\end{fulllineitems}



\subparagraph{LDS.LDS.filters.wave\_filtering\_siso\_abs module}
\label{\detokenize{LDS.LDS.filters:module-LDS.LDS.filters.wave_filtering_siso_abs}}\label{\detokenize{LDS.LDS.filters:lds-lds-filters-wave-filtering-siso-abs-module}}\index{module@\spxentry{module}!LDS.LDS.filters.wave\_filtering\_siso\_abs@\spxentry{LDS.LDS.filters.wave\_filtering\_siso\_abs}}\index{LDS.LDS.filters.wave\_filtering\_siso\_abs@\spxentry{LDS.LDS.filters.wave\_filtering\_siso\_abs}!module@\spxentry{module}}
\sphinxAtStartPar
In FilteringSiso we separated KalmanFilteringSISO. By doing this we dedicated
the written below class to be the abstract class for spectral and persistent filters.
\index{WaveFilteringSisoAbs (class in LDS.LDS.filters.wave\_filtering\_siso\_abs)@\spxentry{WaveFilteringSisoAbs}\spxextra{class in LDS.LDS.filters.wave\_filtering\_siso\_abs}}

\begin{fulllineitems}
\phantomsection\label{\detokenize{LDS.LDS.filters:LDS.LDS.filters.wave_filtering_siso_abs.WaveFilteringSisoAbs}}\pysiglinewithargsret{\sphinxbfcode{\sphinxupquote{class }}\sphinxcode{\sphinxupquote{LDS.LDS.filters.wave\_filtering\_siso\_abs.}}\sphinxbfcode{\sphinxupquote{WaveFilteringSisoAbs}}}{\emph{\DUrole{n}{sys}}, \emph{\DUrole{n}{t\_t}}, \emph{\DUrole{n}{k}}}{}
\sphinxAtStartPar
Bases: {\hyperref[\detokenize{LDS.LDS.filters:LDS.LDS.filters.filtering_siso.FilteringSiso}]{\sphinxcrossref{\sphinxcode{\sphinxupquote{LDS.LDS.filters.filtering\_siso.FilteringSiso}}}}}

\sphinxAtStartPar
Abstract class for creation of persistent and spectral filters.
The subclass WaveFilteringSISO is spectral filter only for symmetric transition matrix.
The related work is “Learning Linear Dynamical Systems via Spectral Filtering”
by E.Hazan, K.Singh and C.Zhang.

\sphinxAtStartPar
WaveFilteringSisoFtl is the class for general case prediction.
The related work is “Spectral Filtering for General Linear Dynamical Systems”
by E.Hazan, K.Singh, H.Lee and C.Zhang.

\sphinxAtStartPar
Hierarchy tree ((ABC)):

\noindent\sphinxincludegraphics{{asciiart-2b351840c3bc463ea62ff24bbf7ff04fb7d27ac3}.png}

\sphinxAtStartPar
Inherits FilteringSiso method.
\begin{quote}\begin{description}
\item[{Parameters}] \leavevmode\begin{itemize}
\item {} 
\sphinxAtStartPar
\sphinxstyleliteralstrong{\sphinxupquote{sys}} \textendash{} LDS. DynamicalSystem object.

\item {} 
\sphinxAtStartPar
\sphinxstyleliteralstrong{\sphinxupquote{t\_t}} \textendash{} Time horizon.

\item {} 
\sphinxAtStartPar
\sphinxstyleliteralstrong{\sphinxupquote{k}} \textendash{} Number of wave\sphinxhyphen{}filters for a spectral filter.

\end{itemize}

\end{description}\end{quote}
\index{predict() (LDS.LDS.filters.wave\_filtering\_siso\_abs.WaveFilteringSisoAbs method)@\spxentry{predict()}\spxextra{LDS.LDS.filters.wave\_filtering\_siso\_abs.WaveFilteringSisoAbs method}}

\begin{fulllineitems}
\phantomsection\label{\detokenize{LDS.LDS.filters:LDS.LDS.filters.wave_filtering_siso_abs.WaveFilteringSisoAbs.predict}}\pysiglinewithargsret{\sphinxbfcode{\sphinxupquote{abstract }}\sphinxbfcode{\sphinxupquote{predict}}}{}{}
\sphinxAtStartPar
Abstract method for calculating output predictions and errors.
\begin{quote}\begin{description}
\item[{Returns}] \leavevmode
\sphinxAtStartPar

\sphinxAtStartPar
tuple containing:
\begin{itemize}
\item {} 
\sphinxAtStartPar
y\_pred\_full           : Output prediction.

\item {} \begin{description}
\item[{M}] \leavevmode{[}Matrix specifying a linear map from featurized inputs to predictions.{]}
\sphinxAtStartPar
Siso filter parameter.

\end{description}

\item {} 
\sphinxAtStartPar
pred\_error            : Spectral filter prediction error.

\item {} 
\sphinxAtStartPar
pred\_error\_persistent : Persistent filter error.

\end{itemize}


\item[{Return type}] \leavevmode
\sphinxAtStartPar
(tuple)

\end{description}\end{quote}

\end{fulllineitems}

\index{var\_calc() (LDS.LDS.filters.wave\_filtering\_siso\_abs.WaveFilteringSisoAbs method)@\spxentry{var\_calc()}\spxextra{LDS.LDS.filters.wave\_filtering\_siso\_abs.WaveFilteringSisoAbs method}}

\begin{fulllineitems}
\phantomsection\label{\detokenize{LDS.LDS.filters:LDS.LDS.filters.wave_filtering_siso_abs.WaveFilteringSisoAbs.var_calc}}\pysiglinewithargsret{\sphinxbfcode{\sphinxupquote{var\_calc}}}{}{}
\sphinxAtStartPar
Initializes spectral filter’s parameters:
\begin{quote}\begin{description}
\item[{Parameters}] \leavevmode\begin{itemize}
\item {} 
\sphinxAtStartPar
\sphinxstyleliteralstrong{\sphinxupquote{n}} \textendash{} Input vector. Shape of processing noise.

\item {} 
\sphinxAtStartPar
\sphinxstyleliteralstrong{\sphinxupquote{m}} \textendash{} Observation vector. Shape of observational error.

\item {} 
\sphinxAtStartPar
\sphinxstyleliteralstrong{\sphinxupquote{k\_dash}} \textendash{} Siso filter parameter.

\item {} 
\sphinxAtStartPar
\sphinxstyleliteralstrong{\sphinxupquote{H}} \textendash{} Hankel matrix.

\item {} 
\sphinxAtStartPar
\sphinxstyleliteralstrong{\sphinxupquote{M}} \textendash{} Matrix specifying a linear map from featurized inputs to predictions.
Siso filter parameter.

\end{itemize}

\end{description}\end{quote}

\end{fulllineitems}


\end{fulllineitems}



\subparagraph{LDS.LDS.filters.wave\_filtering\_siso\_ftl module}
\label{\detokenize{LDS.LDS.filters:module-LDS.LDS.filters.wave_filtering_siso_ftl}}\label{\detokenize{LDS.LDS.filters:lds-lds-filters-wave-filtering-siso-ftl-module}}\index{module@\spxentry{module}!LDS.LDS.filters.wave\_filtering\_siso\_ftl@\spxentry{LDS.LDS.filters.wave\_filtering\_siso\_ftl}}\index{LDS.LDS.filters.wave\_filtering\_siso\_ftl@\spxentry{LDS.LDS.filters.wave\_filtering\_siso\_ftl}!module@\spxentry{module}}
\sphinxAtStartPar
Implements spectral filtering class with follow\sphinxhyphen{}the\sphinxhyphen{}leader algorithm.
Originates from function wave\_filtering\_SISO\_ftl from onlinelds.py.
The related work is “Spectral Filtering for General Linear Dynamical Systems”
by E.Hazan, K.Singh, H.Lee and C.Zhang.
\index{WaveFilteringSisoFtl (class in LDS.LDS.filters.wave\_filtering\_siso\_ftl)@\spxentry{WaveFilteringSisoFtl}\spxextra{class in LDS.LDS.filters.wave\_filtering\_siso\_ftl}}

\begin{fulllineitems}
\phantomsection\label{\detokenize{LDS.LDS.filters:LDS.LDS.filters.wave_filtering_siso_ftl.WaveFilteringSisoFtl}}\pysiglinewithargsret{\sphinxbfcode{\sphinxupquote{class }}\sphinxcode{\sphinxupquote{LDS.LDS.filters.wave\_filtering\_siso\_ftl.}}\sphinxbfcode{\sphinxupquote{WaveFilteringSisoFtl}}}{\emph{\DUrole{n}{sys}}, \emph{\DUrole{n}{t\_t}}, \emph{\DUrole{n}{k}}}{}
\sphinxAtStartPar
Bases: {\hyperref[\detokenize{LDS.LDS.filters:LDS.LDS.filters.wave_filtering_siso_abs.WaveFilteringSisoAbs}]{\sphinxcrossref{\sphinxcode{\sphinxupquote{LDS.LDS.filters.wave\_filtering\_siso\_abs.WaveFilteringSisoAbs}}}}}

\sphinxAtStartPar
Spectral filter with follow\sphinxhyphen{}the\sphinxhyphen{}leader algorithm.

\sphinxAtStartPar
Hierarchy tree ((ABC)):

\noindent\sphinxincludegraphics{{asciiart-2b351840c3bc463ea62ff24bbf7ff04fb7d27ac3}.png}

\sphinxAtStartPar
Inherits all the attributes of its superclass(see WaveFilteringSisoAbs).
With initialization goes through all the methods and gets the predictions.
\begin{quote}\begin{description}
\item[{Parameters}] \leavevmode\begin{itemize}
\item {} 
\sphinxAtStartPar
\sphinxstyleliteralstrong{\sphinxupquote{sys}} \textendash{} LDS. DynamicalSystem object.

\item {} 
\sphinxAtStartPar
\sphinxstyleliteralstrong{\sphinxupquote{t\_t}} \textendash{} Time horizon.

\item {} 
\sphinxAtStartPar
\sphinxstyleliteralstrong{\sphinxupquote{k}} \textendash{} Number of wave\sphinxhyphen{}filters for a spectral filter.

\end{itemize}

\end{description}\end{quote}

\sphinxAtStartPar
Variables initialized with var\_calc():
\begin{quote}\begin{description}
\item[{Variables}] \leavevmode\begin{itemize}
\item {} 
\sphinxAtStartPar
\sphinxstylestrong{n} \textendash{} Input vector. Shape of processing noise.

\item {} 
\sphinxAtStartPar
\sphinxstylestrong{m} \textendash{} Observation vector. Shape of observational error.

\item {} 
\sphinxAtStartPar
\sphinxstylestrong{k\_dash} \textendash{} Siso filter parameter.

\item {} 
\sphinxAtStartPar
\sphinxstylestrong{H} \textendash{} Hankel matrix.

\item {} 
\sphinxAtStartPar
\sphinxstylestrong{M} \textendash{} Matrix specifying a linear map from featurized inputs to predictions.
Siso filter parameter.

\end{itemize}

\end{description}\end{quote}

\sphinxAtStartPar
Uses method args4ftl\_calc to create an array with m and k\_dash.
\index{args4ftl\_calc() (LDS.LDS.filters.wave\_filtering\_siso\_ftl.WaveFilteringSisoFtl method)@\spxentry{args4ftl\_calc()}\spxextra{LDS.LDS.filters.wave\_filtering\_siso\_ftl.WaveFilteringSisoFtl method}}

\begin{fulllineitems}
\phantomsection\label{\detokenize{LDS.LDS.filters:LDS.LDS.filters.wave_filtering_siso_ftl.WaveFilteringSisoFtl.args4ftl_calc}}\pysiglinewithargsret{\sphinxbfcode{\sphinxupquote{args4ftl\_calc}}}{}{}
\sphinxAtStartPar
Parameters calculation.

\sphinxAtStartPar
Creates a 5\sphinxhyphen{}element array with m
on the zero position and k\_dash on the first position.
All others are zeros.
\begin{itemize}
\item {} 
\sphinxAtStartPar
self.m      : Observation vector. Shape of observational error.

\item {} 
\sphinxAtStartPar
self.k\_dash : Siso filter parameter.

\end{itemize}

\end{fulllineitems}

\index{predict() (LDS.LDS.filters.wave\_filtering\_siso\_ftl.WaveFilteringSisoFtl method)@\spxentry{predict()}\spxextra{LDS.LDS.filters.wave\_filtering\_siso\_ftl.WaveFilteringSisoFtl method}}

\begin{fulllineitems}
\phantomsection\label{\detokenize{LDS.LDS.filters:LDS.LDS.filters.wave_filtering_siso_ftl.WaveFilteringSisoFtl.predict}}\pysiglinewithargsret{\sphinxbfcode{\sphinxupquote{predict}}}{}{}
\sphinxAtStartPar
Prediction step.
\begin{quote}\begin{description}
\item[{Returns}] \leavevmode
\sphinxAtStartPar

\sphinxAtStartPar
tuple containing:
\begin{itemize}
\item {} 
\sphinxAtStartPar
y\_pred\_full           : Output prediction.

\item {} \begin{description}
\item[{M}] \leavevmode{[}Matrix specifying a linear map from featurized inputs{]}
\sphinxAtStartPar
to predictions. Siso filter parameter.

\end{description}

\item {} 
\sphinxAtStartPar
pred\_error\_persistent : Persistent filter prediction error.

\end{itemize}


\item[{Return type}] \leavevmode
\sphinxAtStartPar
(tuple)

\end{description}\end{quote}

\end{fulllineitems}


\end{fulllineitems}



\subparagraph{LDS.LDS.filters.wave\_filtering\_siso\_ftl\_persistent module}
\label{\detokenize{LDS.LDS.filters:module-LDS.LDS.filters.wave_filtering_siso_ftl_persistent}}\label{\detokenize{LDS.LDS.filters:lds-lds-filters-wave-filtering-siso-ftl-persistent-module}}\index{module@\spxentry{module}!LDS.LDS.filters.wave\_filtering\_siso\_ftl\_persistent@\spxentry{LDS.LDS.filters.wave\_filtering\_siso\_ftl\_persistent}}\index{LDS.LDS.filters.wave\_filtering\_siso\_ftl\_persistent@\spxentry{LDS.LDS.filters.wave\_filtering\_siso\_ftl\_persistent}!module@\spxentry{module}}
\sphinxAtStartPar
Implements persistent filter with follow\sphinxhyphen{}the\sphinxhyphen{}leader algorithm.
Originates from function wave\_filtering\_SISO\_ftl from onlinelds.py.
The related work is “Spectral Filtering for General Linear Dynamical Systems”
by E.Hazan, K.Singh, H.Lee and C.Zhang.
\index{WaveFilteringSisoFtlPersistent (class in LDS.LDS.filters.wave\_filtering\_siso\_ftl\_persistent)@\spxentry{WaveFilteringSisoFtlPersistent}\spxextra{class in LDS.LDS.filters.wave\_filtering\_siso\_ftl\_persistent}}

\begin{fulllineitems}
\phantomsection\label{\detokenize{LDS.LDS.filters:LDS.LDS.filters.wave_filtering_siso_ftl_persistent.WaveFilteringSisoFtlPersistent}}\pysiglinewithargsret{\sphinxbfcode{\sphinxupquote{class }}\sphinxcode{\sphinxupquote{LDS.LDS.filters.wave\_filtering\_siso\_ftl\_persistent.}}\sphinxbfcode{\sphinxupquote{WaveFilteringSisoFtlPersistent}}}{\emph{\DUrole{n}{sys}}, \emph{\DUrole{n}{t\_t}}, \emph{\DUrole{n}{k}}}{}
\sphinxAtStartPar
Bases: {\hyperref[\detokenize{LDS.LDS.filters:LDS.LDS.filters.wave_filtering_siso_abs.WaveFilteringSisoAbs}]{\sphinxcrossref{\sphinxcode{\sphinxupquote{LDS.LDS.filters.wave\_filtering\_siso\_abs.WaveFilteringSisoAbs}}}}}

\sphinxAtStartPar
Persistent filter with follow\sphinxhyphen{}the\sphinxhyphen{}leader algorithm.

\sphinxAtStartPar
Hierarchy tree ((ABC)):

\noindent\sphinxincludegraphics{{asciiart-3af108a888e1d95cf9dbd3b73b3bbfc8d0307abf}.png}

\sphinxAtStartPar
Inherits all the attributes of its superclass(see WaveFilteringSisoAbs).
With initialization goes through all the methods and gets the predictions.
\begin{quote}\begin{description}
\item[{Parameters}] \leavevmode\begin{itemize}
\item {} 
\sphinxAtStartPar
\sphinxstyleliteralstrong{\sphinxupquote{sys}} \textendash{} LDS. DynamicalSystem object.

\item {} 
\sphinxAtStartPar
\sphinxstyleliteralstrong{\sphinxupquote{t\_t}} \textendash{} Time horizon.

\item {} 
\sphinxAtStartPar
\sphinxstyleliteralstrong{\sphinxupquote{k}} \textendash{} Number of wave\sphinxhyphen{}filters for a spectral filter.

\end{itemize}

\end{description}\end{quote}

\sphinxAtStartPar
Variables initialized with var\_calc():
\begin{quote}\begin{description}
\item[{Variables}] \leavevmode\begin{itemize}
\item {} 
\sphinxAtStartPar
\sphinxstylestrong{n} \textendash{} Input vector. Shape of processing noise.

\item {} 
\sphinxAtStartPar
\sphinxstylestrong{m} \textendash{} Observation vector. Shape of observational error.

\item {} 
\sphinxAtStartPar
\sphinxstylestrong{k\_dash} \textendash{} Siso filter parameter.

\item {} 
\sphinxAtStartPar
\sphinxstylestrong{H} \textendash{} Hankel matrix.

\item {} 
\sphinxAtStartPar
\sphinxstylestrong{M} \textendash{} Matrix specifying a linear map from featurized inputs to predictions.
Siso filter parameter.

\end{itemize}

\end{description}\end{quote}

\sphinxAtStartPar
Uses method args4ftl\_calc to create an array with m and k\_dash.
\index{args4ftl\_calc() (LDS.LDS.filters.wave\_filtering\_siso\_ftl\_persistent.WaveFilteringSisoFtlPersistent method)@\spxentry{args4ftl\_calc()}\spxextra{LDS.LDS.filters.wave\_filtering\_siso\_ftl\_persistent.WaveFilteringSisoFtlPersistent method}}

\begin{fulllineitems}
\phantomsection\label{\detokenize{LDS.LDS.filters:LDS.LDS.filters.wave_filtering_siso_ftl_persistent.WaveFilteringSisoFtlPersistent.args4ftl_calc}}\pysiglinewithargsret{\sphinxbfcode{\sphinxupquote{args4ftl\_calc}}}{}{}
\sphinxAtStartPar
Parameters calculation.

\sphinxAtStartPar
Creates a 5\sphinxhyphen{}element array with m
on the zero position and k\_dash on the first position.
All others are zeros.
\begin{itemize}
\item {} 
\sphinxAtStartPar
self.m      : Observation vector. Shape of observational error.

\item {} 
\sphinxAtStartPar
self.k\_dash : Siso filter parameter.

\end{itemize}

\end{fulllineitems}

\index{predict() (LDS.LDS.filters.wave\_filtering\_siso\_ftl\_persistent.WaveFilteringSisoFtlPersistent method)@\spxentry{predict()}\spxextra{LDS.LDS.filters.wave\_filtering\_siso\_ftl\_persistent.WaveFilteringSisoFtlPersistent method}}

\begin{fulllineitems}
\phantomsection\label{\detokenize{LDS.LDS.filters:LDS.LDS.filters.wave_filtering_siso_ftl_persistent.WaveFilteringSisoFtlPersistent.predict}}\pysiglinewithargsret{\sphinxbfcode{\sphinxupquote{predict}}}{}{}
\sphinxAtStartPar
Prediction step.
\begin{quote}\begin{description}
\item[{Returns}] \leavevmode
\sphinxAtStartPar

\sphinxAtStartPar
tuple containing:
\begin{itemize}
\item {} 
\sphinxAtStartPar
y\_pred\_full           : Output prediction.

\item {} \begin{description}
\item[{M}] \leavevmode{[}Matrix specifying a linear map from featurized inputs{]}
\sphinxAtStartPar
to predictions. Siso filter parameter.

\end{description}

\item {} 
\sphinxAtStartPar
pred\_error\_persistent : Persistent filter prediction error.

\end{itemize}


\item[{Return type}] \leavevmode
\sphinxAtStartPar
(tuple)

\end{description}\end{quote}

\end{fulllineitems}


\end{fulllineitems}



\subparagraph{LDS.LDS.filters.wave\_filtering\_siso\_persistent module}
\label{\detokenize{LDS.LDS.filters:module-LDS.LDS.filters.wave_filtering_siso_persistent}}\label{\detokenize{LDS.LDS.filters:lds-lds-filters-wave-filtering-siso-persistent-module}}\index{module@\spxentry{module}!LDS.LDS.filters.wave\_filtering\_siso\_persistent@\spxentry{LDS.LDS.filters.wave\_filtering\_siso\_persistent}}\index{LDS.LDS.filters.wave\_filtering\_siso\_persistent@\spxentry{LDS.LDS.filters.wave\_filtering\_siso\_persistent}!module@\spxentry{module}}
\sphinxAtStartPar
Originates from function wave\_filtering\_SISO from onlineLDS.py.
The related work is “Learning Linear Dynamical Systems via Spectral Filtering”
by E.Hazan, K.Singh and C.Zhang.
\index{WaveFilteringSISOPersistent (class in LDS.LDS.filters.wave\_filtering\_siso\_persistent)@\spxentry{WaveFilteringSISOPersistent}\spxextra{class in LDS.LDS.filters.wave\_filtering\_siso\_persistent}}

\begin{fulllineitems}
\phantomsection\label{\detokenize{LDS.LDS.filters:LDS.LDS.filters.wave_filtering_siso_persistent.WaveFilteringSISOPersistent}}\pysiglinewithargsret{\sphinxbfcode{\sphinxupquote{class }}\sphinxcode{\sphinxupquote{LDS.LDS.filters.wave\_filtering\_siso\_persistent.}}\sphinxbfcode{\sphinxupquote{WaveFilteringSISOPersistent}}}{\emph{\DUrole{n}{sys}}, \emph{\DUrole{n}{t\_t}}, \emph{\DUrole{n}{k}}, \emph{\DUrole{n}{eta}}, \emph{\DUrole{n}{r\_m}}}{}
\sphinxAtStartPar
Bases: {\hyperref[\detokenize{LDS.LDS.filters:LDS.LDS.filters.wave_filtering_siso_abs.WaveFilteringSisoAbs}]{\sphinxcrossref{\sphinxcode{\sphinxupquote{LDS.LDS.filters.wave\_filtering\_siso\_abs.WaveFilteringSisoAbs}}}}}

\sphinxAtStartPar
Implements persistent filter. \hyperref[\detokenize{LDS.LDS.filters:persistent}]{Fig.\@ \ref{\detokenize{LDS.LDS.filters:persistent}}} shows comparison of filter’s errors.
\hyperref[\detokenize{LDS.LDS.filters:loss}]{Fig.\@ \ref{\detokenize{LDS.LDS.filters:loss}}} shows comparison of spectral and persistent filters.

\sphinxAtStartPar
Hierarchy tree ((ABC)):

\noindent\sphinxincludegraphics{{asciiart-2b351840c3bc463ea62ff24bbf7ff04fb7d27ac3}.png}

\begin{figure}[htbp]
\centering
\capstart

\noindent\sphinxincludegraphics{{persist}.png}
\caption{The errors of the persistent, spectral filters and AR(2) on the first T=100 elements of the
time series.}\label{\detokenize{LDS.LDS.filters:id3}}\label{\detokenize{LDS.LDS.filters:persistent}}\end{figure}

\begin{figure}[htbp]
\centering
\capstart

\noindent\sphinxincludegraphics{{loss}.png}
\caption{The loss of the persistent filter compared to the spectral filter over N=100 iterations.}\label{\detokenize{LDS.LDS.filters:id4}}\label{\detokenize{LDS.LDS.filters:loss}}\end{figure}

\sphinxAtStartPar
Inits all the attributes of its superclass(see WaveFilteringSisoAbs) and
adds Learning rate and Radius parameter.
Goes through all the methods and gets the predictions.
\begin{quote}\begin{description}
\item[{Parameters}] \leavevmode\begin{itemize}
\item {} 
\sphinxAtStartPar
\sphinxstyleliteralstrong{\sphinxupquote{sys}} \textendash{} LDS. DynamicalSystem object.

\item {} 
\sphinxAtStartPar
\sphinxstyleliteralstrong{\sphinxupquote{t\_t}} \textendash{} Time horizon.

\item {} 
\sphinxAtStartPar
\sphinxstyleliteralstrong{\sphinxupquote{k}} \textendash{} Number of wave\sphinxhyphen{}filters for a spectral filter.

\item {} 
\sphinxAtStartPar
\sphinxstyleliteralstrong{\sphinxupquote{eta}} \textendash{} Learning rate.

\item {} 
\sphinxAtStartPar
\sphinxstyleliteralstrong{\sphinxupquote{r\_m}} \textendash{} Radius parameter.

\end{itemize}

\end{description}\end{quote}
\index{predict() (LDS.LDS.filters.wave\_filtering\_siso\_persistent.WaveFilteringSISOPersistent method)@\spxentry{predict()}\spxextra{LDS.LDS.filters.wave\_filtering\_siso\_persistent.WaveFilteringSISOPersistent method}}

\begin{fulllineitems}
\phantomsection\label{\detokenize{LDS.LDS.filters:LDS.LDS.filters.wave_filtering_siso_persistent.WaveFilteringSISOPersistent.predict}}\pysiglinewithargsret{\sphinxbfcode{\sphinxupquote{predict}}}{}{}
\sphinxAtStartPar
Calculation of output predictions and prediction errors.
\begin{quote}\begin{description}
\item[{Returns}] \leavevmode
\sphinxAtStartPar

\sphinxAtStartPar
tuple containing:
\begin{itemize}
\item {} 
\sphinxAtStartPar
y\_pred\_full: Prediction values.

\item {} \begin{description}
\item[{M: Matrix specifying a linear map}] \leavevmode
\sphinxAtStartPar
from featurized inputs to predictions.

\end{description}

\item {} 
\sphinxAtStartPar
pred\_error\_persistent : Persistent filter error.

\end{itemize}


\item[{Return type}] \leavevmode
\sphinxAtStartPar
(tuple)

\end{description}\end{quote}

\end{fulllineitems}


\end{fulllineitems}



\subparagraph{Module contents}
\label{\detokenize{LDS.LDS.filters:module-LDS.LDS.filters}}\label{\detokenize{LDS.LDS.filters:module-contents}}\index{module@\spxentry{module}!LDS.LDS.filters@\spxentry{LDS.LDS.filters}}\index{LDS.LDS.filters@\spxentry{LDS.LDS.filters}!module@\spxentry{module}}

\subparagraph{LDS.LDS.h\_m package}
\label{\detokenize{LDS.LDS.h_m:lds-lds-h-m-package}}\label{\detokenize{LDS.LDS.h_m::doc}}

\subparagraph{Submodules}
\label{\detokenize{LDS.LDS.h_m:submodules}}

\subparagraph{LDS.LDS.h\_m.hankel module}
\label{\detokenize{LDS.LDS.h_m:module-LDS.LDS.h_m.hankel}}\label{\detokenize{LDS.LDS.h_m:lds-lds-h-m-hankel-module}}\index{module@\spxentry{module}!LDS.LDS.h\_m.hankel@\spxentry{LDS.LDS.h\_m.hankel}}\index{LDS.LDS.h\_m.hankel@\spxentry{LDS.LDS.h\_m.hankel}!module@\spxentry{module}}
\sphinxAtStartPar
Implements Hankel matrix.
\index{Hankel (class in LDS.LDS.h\_m.hankel)@\spxentry{Hankel}\spxextra{class in LDS.LDS.h\_m.hankel}}

\begin{fulllineitems}
\phantomsection\label{\detokenize{LDS.LDS.h_m:LDS.LDS.h_m.hankel.Hankel}}\pysiglinewithargsret{\sphinxbfcode{\sphinxupquote{class }}\sphinxcode{\sphinxupquote{LDS.LDS.h\_m.hankel.}}\sphinxbfcode{\sphinxupquote{Hankel}}}{\emph{\DUrole{n}{t\_t}}}{}
\sphinxAtStartPar
Bases: \sphinxcode{\sphinxupquote{object}}

\sphinxAtStartPar
Class originated from onlinelds.py, which
was the first version of the algorithm.
Creates Hankel matrix.

\sphinxAtStartPar
Inits Hankel class with t\_t argument.
Stores Hankel matrix, its eigenvalues
and normalized eigenvectors.
\begin{quote}\begin{description}
\item[{Parameters}] \leavevmode
\sphinxAtStartPar
\sphinxstyleliteralstrong{\sphinxupquote{t\_t}} \textendash{} integer, size of Hankel matrix

\end{description}\end{quote}

\end{fulllineitems}



\subparagraph{Module contents}
\label{\detokenize{LDS.LDS.h_m:module-LDS.LDS.h_m}}\label{\detokenize{LDS.LDS.h_m:module-contents}}\index{module@\spxentry{module}!LDS.LDS.h\_m@\spxentry{LDS.LDS.h\_m}}\index{LDS.LDS.h\_m@\spxentry{LDS.LDS.h\_m}!module@\spxentry{module}}

\subparagraph{LDS.LDS.matlab\_options package}
\label{\detokenize{LDS.LDS.matlab_options:lds-lds-matlab-options-package}}\label{\detokenize{LDS.LDS.matlab_options::doc}}

\subparagraph{Submodules}
\label{\detokenize{LDS.LDS.matlab_options:submodules}}

\subparagraph{LDS.LDS.matlab\_options.matlab\_class\_options module}
\label{\detokenize{LDS.LDS.matlab_options:module-LDS.LDS.matlab_options.matlab_class_options}}\label{\detokenize{LDS.LDS.matlab_options:lds-lds-matlab-options-matlab-class-options-module}}\index{module@\spxentry{module}!LDS.LDS.matlab\_options.matlab\_class\_options@\spxentry{LDS.LDS.matlab\_options.matlab\_class\_options}}\index{LDS.LDS.matlab\_options.matlab\_class\_options@\spxentry{LDS.LDS.matlab\_options.matlab\_class\_options}!module@\spxentry{module}}
\sphinxAtStartPar
Analogy of Matlab options
\index{ClassOptions (class in LDS.LDS.matlab\_options.matlab\_class\_options)@\spxentry{ClassOptions}\spxextra{class in LDS.LDS.matlab\_options.matlab\_class\_options}}

\begin{fulllineitems}
\phantomsection\label{\detokenize{LDS.LDS.matlab_options:LDS.LDS.matlab_options.matlab_class_options.ClassOptions}}\pysigline{\sphinxbfcode{\sphinxupquote{class }}\sphinxcode{\sphinxupquote{LDS.LDS.matlab\_options.matlab\_class\_options.}}\sphinxbfcode{\sphinxupquote{ClassOptions}}}
\sphinxAtStartPar
Bases: \sphinxcode{\sphinxupquote{object}}

\sphinxAtStartPar
Mimics ‘options’ from Matlab

\end{fulllineitems}



\subparagraph{Module contents}
\label{\detokenize{LDS.LDS.matlab_options:module-LDS.LDS.matlab_options}}\label{\detokenize{LDS.LDS.matlab_options:module-contents}}\index{module@\spxentry{module}!LDS.LDS.matlab\_options@\spxentry{LDS.LDS.matlab\_options}}\index{LDS.LDS.matlab\_options@\spxentry{LDS.LDS.matlab\_options}!module@\spxentry{module}}

\subparagraph{LDS.LDS.online\_lds package}
\label{\detokenize{LDS.LDS.online_lds:lds-lds-online-lds-package}}\label{\detokenize{LDS.LDS.online_lds::doc}}

\subparagraph{Submodules}
\label{\detokenize{LDS.LDS.online_lds:submodules}}

\subparagraph{LDS.LDS.online\_lds.cost\_ftl module}
\label{\detokenize{LDS.LDS.online_lds:module-LDS.LDS.online_lds.cost_ftl}}\label{\detokenize{LDS.LDS.online_lds:lds-lds-online-lds-cost-ftl-module}}\index{module@\spxentry{module}!LDS.LDS.online\_lds.cost\_ftl@\spxentry{LDS.LDS.online\_lds.cost\_ftl}}\index{LDS.LDS.online\_lds.cost\_ftl@\spxentry{LDS.LDS.online\_lds.cost\_ftl}!module@\spxentry{module}}\index{cost\_ftl() (in module LDS.LDS.online\_lds.cost\_ftl)@\spxentry{cost\_ftl()}\spxextra{in module LDS.LDS.online\_lds.cost\_ftl}}

\begin{fulllineitems}
\phantomsection\label{\detokenize{LDS.LDS.online_lds:LDS.LDS.online_lds.cost_ftl.cost_ftl}}\pysiglinewithargsret{\sphinxcode{\sphinxupquote{LDS.LDS.online\_lds.cost\_ftl.}}\sphinxbfcode{\sphinxupquote{cost\_ftl}}}{\emph{\DUrole{n}{M\_flat}}, \emph{\DUrole{o}{*}\DUrole{n}{args}}}{}
\sphinxAtStartPar
from onlinelds.py

\end{fulllineitems}



\subparagraph{LDS.LDS.online\_lds.gradient\_ftl module}
\label{\detokenize{LDS.LDS.online_lds:module-LDS.LDS.online_lds.gradient_ftl}}\label{\detokenize{LDS.LDS.online_lds:lds-lds-online-lds-gradient-ftl-module}}\index{module@\spxentry{module}!LDS.LDS.online\_lds.gradient\_ftl@\spxentry{LDS.LDS.online\_lds.gradient\_ftl}}\index{LDS.LDS.online\_lds.gradient\_ftl@\spxentry{LDS.LDS.online\_lds.gradient\_ftl}!module@\spxentry{module}}\index{gradient\_ftl() (in module LDS.LDS.online\_lds.gradient\_ftl)@\spxentry{gradient\_ftl()}\spxextra{in module LDS.LDS.online\_lds.gradient\_ftl}}

\begin{fulllineitems}
\phantomsection\label{\detokenize{LDS.LDS.online_lds:LDS.LDS.online_lds.gradient_ftl.gradient_ftl}}\pysiglinewithargsret{\sphinxcode{\sphinxupquote{LDS.LDS.online\_lds.gradient\_ftl.}}\sphinxbfcode{\sphinxupquote{gradient\_ftl}}}{\emph{\DUrole{n}{M\_flat}}, \emph{\DUrole{o}{*}\DUrole{n}{args}}}{}
\sphinxAtStartPar
from onlinelds.py

\end{fulllineitems}



\subparagraph{LDS.LDS.online\_lds.print\_verbose module}
\label{\detokenize{LDS.LDS.online_lds:module-LDS.LDS.online_lds.print_verbose}}\label{\detokenize{LDS.LDS.online_lds:lds-lds-online-lds-print-verbose-module}}\index{module@\spxentry{module}!LDS.LDS.online\_lds.print\_verbose@\spxentry{LDS.LDS.online\_lds.print\_verbose}}\index{LDS.LDS.online\_lds.print\_verbose@\spxentry{LDS.LDS.online\_lds.print\_verbose}!module@\spxentry{module}}\index{print\_verbose() (in module LDS.LDS.online\_lds.print\_verbose)@\spxentry{print\_verbose()}\spxextra{in module LDS.LDS.online\_lds.print\_verbose}}

\begin{fulllineitems}
\phantomsection\label{\detokenize{LDS.LDS.online_lds:LDS.LDS.online_lds.print_verbose.print_verbose}}\pysiglinewithargsret{\sphinxcode{\sphinxupquote{LDS.LDS.online\_lds.print\_verbose.}}\sphinxbfcode{\sphinxupquote{print\_verbose}}}{\emph{\DUrole{n}{a}}, \emph{\DUrole{n}{verbose}}}{}
\sphinxAtStartPar
from onlinelds.py

\end{fulllineitems}



\subparagraph{Module contents}
\label{\detokenize{LDS.LDS.online_lds:module-LDS.LDS.online_lds}}\label{\detokenize{LDS.LDS.online_lds:module-contents}}\index{module@\spxentry{module}!LDS.LDS.online\_lds@\spxentry{LDS.LDS.online\_lds}}\index{LDS.LDS.online\_lds@\spxentry{LDS.LDS.online\_lds}!module@\spxentry{module}}

\subparagraph{LDS.LDS.ts package}
\label{\detokenize{LDS.LDS.ts:lds-lds-ts-package}}\label{\detokenize{LDS.LDS.ts::doc}}

\subparagraph{Submodules}
\label{\detokenize{LDS.LDS.ts:submodules}}

\subparagraph{LDS.LDS.ts.time\_series module}
\label{\detokenize{LDS.LDS.ts:module-LDS.LDS.ts.time_series}}\label{\detokenize{LDS.LDS.ts:lds-lds-ts-time-series-module}}\index{module@\spxentry{module}!LDS.LDS.ts.time\_series@\spxentry{LDS.LDS.ts.time\_series}}\index{LDS.LDS.ts.time\_series@\spxentry{LDS.LDS.ts.time\_series}!module@\spxentry{module}}
\sphinxAtStartPar
Implements time series from inputlds.py
\index{TimeSeries (class in LDS.LDS.ts.time\_series)@\spxentry{TimeSeries}\spxextra{class in LDS.LDS.ts.time\_series}}

\begin{fulllineitems}
\phantomsection\label{\detokenize{LDS.LDS.ts:LDS.LDS.ts.time_series.TimeSeries}}\pysiglinewithargsret{\sphinxbfcode{\sphinxupquote{class }}\sphinxcode{\sphinxupquote{LDS.LDS.ts.time\_series.}}\sphinxbfcode{\sphinxupquote{TimeSeries}}}{\emph{\DUrole{n}{matlabfile}}, \emph{\DUrole{n}{varname}}}{}
\sphinxAtStartPar
Bases: \sphinxcode{\sphinxupquote{object}}

\sphinxAtStartPar
Class originated from inputlds.py, which
was the first version of the algorithm.

\sphinxAtStartPar
Inits TimeSeries.
\begin{quote}\begin{description}
\item[{Parameters}] \leavevmode\begin{itemize}
\item {} 
\sphinxAtStartPar
\sphinxstyleliteralstrong{\sphinxupquote{matlabfile}} \textendash{} the matlab file ‘./OARIMA\_code\_data/data/setting6.mat’

\item {} 
\sphinxAtStartPar
\sphinxstyleliteralstrong{\sphinxupquote{varname}} \textendash{} uses ‘seq\_d0’. 1 x 35701 double vector.

\end{itemize}

\item[{Raises}] \leavevmode
\sphinxAtStartPar
\sphinxstyleliteralstrong{\sphinxupquote{HDF5ExtError}} \textendash{} can’t open given matlabfile.

\end{description}\end{quote}
\index{logratio() (LDS.LDS.ts.time\_series.TimeSeries method)@\spxentry{logratio()}\spxextra{LDS.LDS.ts.time\_series.TimeSeries method}}

\begin{fulllineitems}
\phantomsection\label{\detokenize{LDS.LDS.ts:LDS.LDS.ts.time_series.TimeSeries.logratio}}\pysiglinewithargsret{\sphinxbfcode{\sphinxupquote{logratio}}}{}{}
\sphinxAtStartPar
Replaces the time series by a log\sphinxhyphen{}ratio of subsequent element therein.

\end{fulllineitems}

\index{solve() (LDS.LDS.ts.time\_series.TimeSeries method)@\spxentry{solve()}\spxextra{LDS.LDS.ts.time\_series.TimeSeries method}}

\begin{fulllineitems}
\phantomsection\label{\detokenize{LDS.LDS.ts:LDS.LDS.ts.time_series.TimeSeries.solve}}\pysiglinewithargsret{\sphinxbfcode{\sphinxupquote{solve}}}{\emph{\DUrole{n}{h\_zero}\DUrole{o}{=}\DUrole{default_value}{{[}{]}}}, \emph{\DUrole{n}{inputs}\DUrole{o}{=}\DUrole{default_value}{{[}{]}}}, \emph{\DUrole{n}{t\_t}\DUrole{o}{=}\DUrole{default_value}{100}}, \emph{\DUrole{o}{**}\DUrole{n}{kwargs}}}{}
\sphinxAtStartPar
This just truncates the series loaded in the constructor.
\begin{quote}\begin{description}
\item[{Raises}] \leavevmode
\sphinxAtStartPar
\sphinxstyleliteralstrong{\sphinxupquote{Exits if time horizon isn\textquotesingle{}t an integer.}} \textendash{} 

\end{description}\end{quote}

\end{fulllineitems}


\end{fulllineitems}



\subparagraph{Module contents}
\label{\detokenize{LDS.LDS.ts:module-LDS.LDS.ts}}\label{\detokenize{LDS.LDS.ts:module-contents}}\index{module@\spxentry{module}!LDS.LDS.ts@\spxentry{LDS.LDS.ts}}\index{LDS.LDS.ts@\spxentry{LDS.LDS.ts}!module@\spxentry{module}}

\paragraph{Module contents}
\label{\detokenize{LDS.LDS:module-LDS.LDS}}\label{\detokenize{LDS.LDS:module-contents}}\index{module@\spxentry{module}!LDS.LDS@\spxentry{LDS.LDS}}\index{LDS.LDS@\spxentry{LDS.LDS}!module@\spxentry{module}}

\subsection{Submodules}
\label{\detokenize{LDS:submodules}}

\subsection{LDS.OnlineLDS\_library module}
\label{\detokenize{LDS:module-LDS.OnlineLDS_library}}\label{\detokenize{LDS:lds-onlinelds-library-module}}\index{module@\spxentry{module}!LDS.OnlineLDS\_library@\spxentry{LDS.OnlineLDS\_library}}\index{LDS.OnlineLDS\_library@\spxentry{LDS.OnlineLDS\_library}!module@\spxentry{module}}\index{A\_trans\_calc() (in module LDS.OnlineLDS\_library)@\spxentry{A\_trans\_calc()}\spxextra{in module LDS.OnlineLDS\_library}}

\begin{fulllineitems}
\phantomsection\label{\detokenize{LDS:LDS.OnlineLDS_library.A_trans_calc}}\pysiglinewithargsret{\sphinxcode{\sphinxupquote{LDS.OnlineLDS\_library.}}\sphinxbfcode{\sphinxupquote{A\_trans\_calc}}}{\emph{\DUrole{n}{A\_trans}}, \emph{\DUrole{n}{grad}}}{}
\sphinxAtStartPar
begin\{gather\}
Regret(w\_\{1:T\}) = sum\_\{t=0\}\textasciicircum{}\{T\}l(  heta\_t)\sphinxhyphen{}min\_\{L in S\} sum\_\{t=0\}\textasciicircum{}\{T\}l(y\_\{t\},f\_\{t\}(L)),
end\{gather\}
MATLAB:
A\_trans = A\_trans \sphinxhyphen{} A\_trans * grad’ * grad * A\_trans/(1 + grad * A\_trans * grad’);
we have to convert data{[}{]} from 1D vector to a numpy matrix (2D) to apply the transpose
OR data{[}{]}.reshape(\sphinxhyphen{}1,1) can be also used to mimick the transpose.
\begin{quote}\begin{description}
\item[{Parameters}] \leavevmode\begin{itemize}
\item {} 
\sphinxAtStartPar
\sphinxstyleliteralstrong{\sphinxupquote{A\_trans}} \textendash{} np.eye(mk) * epsilon

\item {} 
\sphinxAtStartPar
\sphinxstyleliteralstrong{\sphinxupquote{grad}} \textendash{} Gradient, the return of the function grad\_calc.

\end{itemize}

\end{description}\end{quote}

\sphinxAtStartPar
Returns:

\end{fulllineitems}

\index{arima\_ogd() (in module LDS.OnlineLDS\_library)@\spxentry{arima\_ogd()}\spxextra{in module LDS.OnlineLDS\_library}}

\begin{fulllineitems}
\phantomsection\label{\detokenize{LDS:LDS.OnlineLDS_library.arima_ogd}}\pysiglinewithargsret{\sphinxcode{\sphinxupquote{LDS.OnlineLDS\_library.}}\sphinxbfcode{\sphinxupquote{arima\_ogd}}}{\emph{\DUrole{n}{data}}, \emph{\DUrole{n}{options}}}{}
\sphinxAtStartPar
from arima\_ogd.m
Used by example.py. ARIMA Online Newton Step algorithm.
The function was originally written in MATLAB by Liu, C.; Hoi, S. C. H.; Zhao, P.; and Sun, J.
It’s described in their work “Online arima algorithms for time series prediction.”
\begin{quote}\begin{description}
\item[{Parameters}] \leavevmode\begin{itemize}
\item {} 
\sphinxAtStartPar
\sphinxstyleliteralstrong{\sphinxupquote{data}} \textendash{} Array of 10000 elements.

\item {} 
\sphinxAtStartPar
\sphinxstyleliteralstrong{\sphinxupquote{options}} \textendash{} Instance of ClassOptions class.

\end{itemize}

\end{description}\end{quote}

\sphinxAtStartPar
Returns:

\end{fulllineitems}

\index{arima\_ons() (in module LDS.OnlineLDS\_library)@\spxentry{arima\_ons()}\spxextra{in module LDS.OnlineLDS\_library}}

\begin{fulllineitems}
\phantomsection\label{\detokenize{LDS:LDS.OnlineLDS_library.arima_ons}}\pysiglinewithargsret{\sphinxcode{\sphinxupquote{LDS.OnlineLDS\_library.}}\sphinxbfcode{\sphinxupquote{arima\_ons}}}{\emph{\DUrole{n}{data}}, \emph{\DUrole{n}{options}}}{}
\sphinxAtStartPar
Originates from arima\_ons.m. ARIMA Online Newton Step algorithm.
Used by example.py.
The function was originally written in MATLAB by Liu, C.; Hoi, S. C. H.; Zhao, P.; and Sun, J.
It’s described in their work “Online arima algorithms for time series prediction.”
\begin{quote}\begin{description}
\item[{Parameters}] \leavevmode\begin{itemize}
\item {} 
\sphinxAtStartPar
\sphinxstyleliteralstrong{\sphinxupquote{data}} \textendash{} Array of 10000 elements.

\item {} 
\sphinxAtStartPar
\sphinxstyleliteralstrong{\sphinxupquote{options}} \textendash{} Instance of ClassOptions class.

\end{itemize}

\end{description}\end{quote}

\sphinxAtStartPar
Returns:

\end{fulllineitems}

\index{close\_all\_figs() (in module LDS.OnlineLDS\_library)@\spxentry{close\_all\_figs()}\spxextra{in module LDS.OnlineLDS\_library}}

\begin{fulllineitems}
\phantomsection\label{\detokenize{LDS:LDS.OnlineLDS_library.close_all_figs}}\pysiglinewithargsret{\sphinxcode{\sphinxupquote{LDS.OnlineLDS\_library.}}\sphinxbfcode{\sphinxupquote{close\_all\_figs}}}{}{}
\sphinxAtStartPar
Closes all the figures. Originally the function comes from experiments.py file.

\end{fulllineitems}

\index{cost\_ar() (in module LDS.OnlineLDS\_library)@\spxentry{cost\_ar()}\spxextra{in module LDS.OnlineLDS\_library}}

\begin{fulllineitems}
\phantomsection\label{\detokenize{LDS:LDS.OnlineLDS_library.cost_ar}}\pysiglinewithargsret{\sphinxcode{\sphinxupquote{LDS.OnlineLDS\_library.}}\sphinxbfcode{\sphinxupquote{cost\_ar}}}{\emph{\DUrole{n}{theta}}, \emph{\DUrole{o}{*}\DUrole{n}{args}}}{}
\sphinxAtStartPar
Loss function of auto\sphinxhyphen{}regression.
After the prediction is made, the true observation is revealed to
the algorithm, and a loss associated with the prediction is computed.
Here we consider the quadratic loss for simplicity.
Originally the function comes from onlinelds.py file.
\begin{quote}\begin{description}
\item[{Parameters}] \leavevmode\begin{itemize}
\item {} 
\sphinxAtStartPar
\sphinxstyleliteralstrong{\sphinxupquote{theta}} \textendash{} auto\sphinxhyphen{}regressive parameters.

\item {} 
\sphinxAtStartPar
\sphinxstyleliteralstrong{\sphinxupquote{args}}\sphinxstyleliteralstrong{\sphinxupquote{{[}}}\sphinxstyleliteralstrong{\sphinxupquote{0}}\sphinxstyleliteralstrong{\sphinxupquote{{]}}} \textendash{} observation at time t

\item {} 
\sphinxAtStartPar
\sphinxstyleliteralstrong{\sphinxupquote{args}}\sphinxstyleliteralstrong{\sphinxupquote{{[}}}\sphinxstyleliteralstrong{\sphinxupquote{1}}\sphinxstyleliteralstrong{\sphinxupquote{{]}}} \textendash{} past s observations (most to least recent: t\sphinxhyphen{}1 to t\sphinxhyphen{}1\sphinxhyphen{}s)

\end{itemize}

\item[{Returns}] \leavevmode
\sphinxAtStartPar
Quadratic loss function of auto\sphinxhyphen{}regression.

\end{description}\end{quote}

\end{fulllineitems}

\index{diff\_calc() (in module LDS.OnlineLDS\_library)@\spxentry{diff\_calc()}\spxextra{in module LDS.OnlineLDS\_library}}

\begin{fulllineitems}
\phantomsection\label{\detokenize{LDS:LDS.OnlineLDS_library.diff_calc}}\pysiglinewithargsret{\sphinxcode{\sphinxupquote{LDS.OnlineLDS\_library.}}\sphinxbfcode{\sphinxupquote{diff\_calc}}}{\emph{\DUrole{n}{w}}, \emph{\DUrole{n}{data}}, \emph{\DUrole{n}{mk}}, \emph{\DUrole{n}{i}}}{}
\sphinxAtStartPar
Auxiliary function to implement ARIMA in python. Others functions use it in their
iterations.
MATLAB: diff = w*data(i\sphinxhyphen{}mk:i\sphinxhyphen{}1)’\sphinxhyphen{}data(i);
remember! MATLAB\_data(1) == Python\_data{[}0{]}
we have to convert data{[}{]} from 1D vector to a numpy matrix (2D) to apply the transpose
OR data{[}{]}.reshape(\sphinxhyphen{}1,1) can be also used to mimick the transpose
\begin{quote}\begin{description}
\item[{Parameters}] \leavevmode\begin{itemize}
\item {} 
\sphinxAtStartPar
\sphinxstyleliteralstrong{\sphinxupquote{w}} \textendash{} Uniform distribution array with options.mk number of columns.

\item {} 
\sphinxAtStartPar
\sphinxstyleliteralstrong{\sphinxupquote{data}} \textendash{} Array of 10000 elements.

\item {} 
\sphinxAtStartPar
\sphinxstyleliteralstrong{\sphinxupquote{mk}} \textendash{} Integer number. Here we used 10.

\item {} 
\sphinxAtStartPar
\sphinxstyleliteralstrong{\sphinxupquote{i}} \textendash{} Iterative number. In range from mk till data \sphinxhyphen{} 1.

\end{itemize}

\end{description}\end{quote}

\sphinxAtStartPar
Returns:

\end{fulllineitems}

\index{error\_stat() (in module LDS.OnlineLDS\_library)@\spxentry{error\_stat()}\spxextra{in module LDS.OnlineLDS\_library}}

\begin{fulllineitems}
\phantomsection\label{\detokenize{LDS:LDS.OnlineLDS_library.error_stat}}\pysiglinewithargsret{\sphinxcode{\sphinxupquote{LDS.OnlineLDS\_library.}}\sphinxbfcode{\sphinxupquote{error\_stat}}}{\emph{\DUrole{n}{error\_spec\_data}}, \emph{\DUrole{n}{error\_persist\_data}}}{}
\sphinxAtStartPar
if have\_spectral\_persistent:
\begin{quote}\begin{description}
\item[{Returns}] \leavevmode
\sphinxAtStartPar
Mean error of spectral filtering
error\_spec\_std:     Std of spectral filtering error
error\_persist\_mean: Mean error of last\sphinxhyphen{}value prediction
error\_persist\_std:  Std of last\sphinxhyphen{}value prediction error

\item[{Return type}] \leavevmode
\sphinxAtStartPar
error\_spec\_mean

\end{description}\end{quote}

\end{fulllineitems}

\index{grad\_calc() (in module LDS.OnlineLDS\_library)@\spxentry{grad\_calc()}\spxextra{in module LDS.OnlineLDS\_library}}

\begin{fulllineitems}
\phantomsection\label{\detokenize{LDS:LDS.OnlineLDS_library.grad_calc}}\pysiglinewithargsret{\sphinxcode{\sphinxupquote{LDS.OnlineLDS\_library.}}\sphinxbfcode{\sphinxupquote{grad\_calc}}}{\emph{\DUrole{n}{data}}, \emph{\DUrole{n}{i}}, \emph{\DUrole{n}{mk}}, \emph{\DUrole{n}{diff}}}{}
\sphinxAtStartPar
MATLAB: grad = 2*data(i\sphinxhyphen{}mk:i\sphinxhyphen{}1)*diff
Used by function arima\_ons.
\begin{quote}\begin{description}
\item[{Parameters}] \leavevmode\begin{itemize}
\item {} 
\sphinxAtStartPar
\sphinxstyleliteralstrong{\sphinxupquote{data}} \textendash{} Array of 10000 elements.

\item {} 
\sphinxAtStartPar
\sphinxstyleliteralstrong{\sphinxupquote{i}} \textendash{} Iterative number. In range from mk till data \sphinxhyphen{} 1.

\item {} 
\sphinxAtStartPar
\sphinxstyleliteralstrong{\sphinxupquote{mk}} \textendash{} Integer number. Here we used 10.

\item {} 
\sphinxAtStartPar
\sphinxstyleliteralstrong{\sphinxupquote{diff}} \textendash{} Result of diff\_calc function

\end{itemize}

\item[{Returns}] \leavevmode
\sphinxAtStartPar
Gradient.

\end{description}\end{quote}

\end{fulllineitems}

\index{gradient\_ar() (in module LDS.OnlineLDS\_library)@\spxentry{gradient\_ar()}\spxextra{in module LDS.OnlineLDS\_library}}

\begin{fulllineitems}
\phantomsection\label{\detokenize{LDS:LDS.OnlineLDS_library.gradient_ar}}\pysiglinewithargsret{\sphinxcode{\sphinxupquote{LDS.OnlineLDS\_library.}}\sphinxbfcode{\sphinxupquote{gradient\_ar}}}{\emph{\DUrole{n}{theta}}, \emph{\DUrole{o}{*}\DUrole{n}{args}}}{}
\sphinxAtStartPar
Gradient function of auto\sphinxhyphen{}regression.
We use the general scheme of on\sphinxhyphen{}line gradient decent algorithms,
where the update goes against the direction of the gradient of the current loss.
In addition, it is useful to restrict the state to a bounded domain.
Originally the function comes from onlinelds.py file.
\begin{quote}\begin{description}
\item[{Parameters}] \leavevmode\begin{itemize}
\item {} 
\sphinxAtStartPar
\sphinxstyleliteralstrong{\sphinxupquote{theta}} \textendash{} s parameters.

\item {} 
\sphinxAtStartPar
\sphinxstyleliteralstrong{\sphinxupquote{args}}\sphinxstyleliteralstrong{\sphinxupquote{{[}}}\sphinxstyleliteralstrong{\sphinxupquote{0}}\sphinxstyleliteralstrong{\sphinxupquote{{]}}} \textendash{} Observation.

\item {} 
\sphinxAtStartPar
\sphinxstyleliteralstrong{\sphinxupquote{args}}\sphinxstyleliteralstrong{\sphinxupquote{{[}}}\sphinxstyleliteralstrong{\sphinxupquote{1}}\sphinxstyleliteralstrong{\sphinxupquote{{]}}} \textendash{} Past s observations.

\end{itemize}

\item[{Returns}] \leavevmode
\sphinxAtStartPar
Gradient function of auto\sphinxhyphen{}regression.

\end{description}\end{quote}

\end{fulllineitems}

\index{heatmap() (in module LDS.OnlineLDS\_library)@\spxentry{heatmap()}\spxextra{in module LDS.OnlineLDS\_library}}

\begin{fulllineitems}
\phantomsection\label{\detokenize{LDS:LDS.OnlineLDS_library.heatmap}}\pysiglinewithargsret{\sphinxcode{\sphinxupquote{LDS.OnlineLDS\_library.}}\sphinxbfcode{\sphinxupquote{heatmap}}}{\emph{\DUrole{n}{data}}, \emph{\DUrole{n}{row\_labels}}, \emph{\DUrole{n}{col\_labels}}, \emph{\DUrole{n}{ax}\DUrole{o}{=}\DUrole{default_value}{None}}, \emph{\DUrole{n}{cbar\_kw}\DUrole{o}{=}\DUrole{default_value}{\{\}}}, \emph{\DUrole{n}{cbarlabel}\DUrole{o}{=}\DUrole{default_value}{\textquotesingle{}\textquotesingle{}}}, \emph{\DUrole{o}{**}\DUrole{n}{kwargs}}}{}
\sphinxAtStartPar
The function is taken from pyplot documentation.
Create a heatmap from a numpy array and two lists of labels.
Used by testNoiseImpact to implement \hyperref[\detokenize{LDS:fig3}]{Fig.\@ \ref{\detokenize{LDS:fig3}}} and \hyperref[\detokenize{LDS:fig6}]{Fig.\@ \ref{\detokenize{LDS:fig6}}}.
Originally the function comes from experiments.py file.

\begin{figure}[htbp]
\centering
\capstart

\noindent\sphinxincludegraphics{{Figure3}.png}
\caption{The ratio of the errors of Kalman filter and AR(2) on Example 7 from Marecek’s paper
indicated by colours as a function of \(w, v\) of process and observation noise, on
the vertical and horizontal axes, resp. Origin is the top\sphinxhyphen{}left corner.}\label{\detokenize{LDS:id1}}\label{\detokenize{LDS:fig3}}\end{figure}

\begin{figure}[htbp]
\centering
\capstart

\noindent\sphinxincludegraphics{{Figure6}.png}
\caption{The effect of varying the magnitude of noise in Example 7 on AR(2)
(top), AR(4) (middle), and AR(8) (bottom). Left: average RMSE of predictions
of AR(s+ 1) as a function of the variance of the process noise (vertical axis) and
observation noise (horizontal axis). Center: The differences in average RMSE of
Kalman filters and AR(s + 1) as a function of the variance of the process noise
(vertical axis) and observation noise (horizontal axis). Throughout averages are
taken over 10 runs. Right: The ratio (70) of the errors of Kalman filters and
AR(s + 1) as a function of the variance of the process noise (vertical axis) and
observation noise (horizontal axis). Throughout, notice the origin is in the topleft corner.}\label{\detokenize{LDS:id2}}\label{\detokenize{LDS:fig6}}\end{figure}
\begin{quote}\begin{description}
\item[{Parameters}] \leavevmode\begin{itemize}
\item {} 
\sphinxAtStartPar
\sphinxstyleliteralstrong{\sphinxupquote{data}} \textendash{} A 2D numpy array of shape (N,M)

\item {} 
\sphinxAtStartPar
\sphinxstyleliteralstrong{\sphinxupquote{row\_labels}} \textendash{} A list or array of length N with the labels
for the rows

\item {} 
\sphinxAtStartPar
\sphinxstyleliteralstrong{\sphinxupquote{col\_labels}} \textendash{} A list or array of length M with the labels
for the columns

\end{itemize}

\item[{Optional arguments}] \leavevmode\begin{itemize}
\item {} 
\sphinxAtStartPar
\sphinxstylestrong{ax} \textendash{} A matplotlib.axes.Axes instance to which the heatmap
is plotted. If not provided, use current axes or
create a new one.

\item {} 
\sphinxAtStartPar
\sphinxstylestrong{cbar\_kw} \textendash{} A dictionary with arguments to
\sphinxcode{\sphinxupquote{matplotlib.Figure.colorbar()}}.

\item {} 
\sphinxAtStartPar
\sphinxstylestrong{cbarlabel} \textendash{} The label for the colorbar

\end{itemize}

\end{description}\end{quote}

\sphinxAtStartPar
All other arguments are directly passed on to the imshow call.

\end{fulllineitems}

\index{lab() (in module LDS.OnlineLDS\_library)@\spxentry{lab()}\spxextra{in module LDS.OnlineLDS\_library}}

\begin{fulllineitems}
\phantomsection\label{\detokenize{LDS:LDS.OnlineLDS_library.lab}}\pysiglinewithargsret{\sphinxcode{\sphinxupquote{LDS.OnlineLDS\_library.}}\sphinxbfcode{\sphinxupquote{lab}}}{\emph{\DUrole{n}{s}}, \emph{\DUrole{n}{eta\_zero}}}{}
\sphinxAtStartPar
Gives a label to auto\sphinxhyphen{}regression outputs and labels in seq0,seq1,seq2 pdfs.
\begin{quote}\begin{description}
\item[{Returns}] \leavevmode
\sphinxAtStartPar
auto\sphinxhyphen{}regression label. Example: “AR(2), c = 2500”.

\item[{Return type}] \leavevmode
\sphinxAtStartPar
lab1

\end{description}\end{quote}

\end{fulllineitems}

\index{p3\_for\_test\_identification2() (in module LDS.OnlineLDS\_library)@\spxentry{p3\_for\_test\_identification2()}\spxextra{in module LDS.OnlineLDS\_library}}

\begin{fulllineitems}
\phantomsection\label{\detokenize{LDS:LDS.OnlineLDS_library.p3_for_test_identification2}}\pysiglinewithargsret{\sphinxcode{\sphinxupquote{LDS.OnlineLDS\_library.}}\sphinxbfcode{\sphinxupquote{p3\_for\_test\_identification2}}}{\emph{\DUrole{n}{ylim}}, \emph{\DUrole{n}{have\_spectral\_persistent}}, \emph{\DUrole{n}{Tlim}}, \emph{\DUrole{n}{error\_spec}}, \emph{\DUrole{n}{sequence\_label}}, \emph{\DUrole{n}{error\_spec\_mean}}, \emph{\DUrole{n}{error\_spec\_std}}, \emph{\DUrole{n}{alphaValue}}, \emph{\DUrole{n}{error\_persist}}, \emph{\DUrole{n}{error\_persist\_mean}}, \emph{\DUrole{n}{error\_persist\_std}}, \emph{\DUrole{n}{error\_AR1\_mean}}, \emph{\DUrole{n}{error\_AR1\_std}}, \emph{\DUrole{n}{have\_kalman}}, \emph{\DUrole{n}{error\_Kalman\_mean}}, \emph{\DUrole{n}{error\_Kalman\_std}}, \emph{\DUrole{n}{p\_p}}}{}
\sphinxAtStartPar
Plots \hyperref[\detokenize{LDS:fig2}]{Fig.\@ \ref{\detokenize{LDS:fig2}}}, \hyperref[\detokenize{LDS:fig5}]{Fig.\@ \ref{\detokenize{LDS:fig5}}} after getting all the errors data.
In \hyperref[\detokenize{LDS:fig2}]{Fig.\@ \ref{\detokenize{LDS:fig2}}}, we compare the prediction error for 4 methods:
the standard baseline last\sphinxhyphen{}value prediction \(\hat{y}_{t+1} := y_t\), also
known as persistence prediction, the spectral filtering of
\textbackslash{}cite\{hazan2017online\}, Kalman filter, and AR(2).

\sphinxAtStartPar
We first continue the Example \textbackslash{}ref\{HazanEx\} form the main body of the
paper, with a system given by (\textbackslash{}ref\{eq:experem1\_system\_hazan\}) and
\(v=w=0.5\). Figure \textbackslash{}ref\{fig1\}(right) shows a sample observations
trajectory of the system, together with forecast for the four methods.
Figure \textbackslash{}ref\{fig1\}(left) show the mean and standard deviations of the
errors for the first 500 time steps. Figure \textbackslash{}ref\{fig1brief\} in the main
text is the restriction of this Figure \textbackslash{}ref\{fig1\}(left) to the first 20
steps. Similarly to Figure \textbackslash{}ref\{fig1brief\}, we observe that the AR(2)
predictions are better than the spectral and persistence methods, and
worse than the Kalman filter, since only two first terms are considered.

\end{fulllineitems}

\index{plot\_p1() (in module LDS.OnlineLDS\_library)@\spxentry{plot\_p1()}\spxextra{in module LDS.OnlineLDS\_library}}

\begin{fulllineitems}
\phantomsection\label{\detokenize{LDS:LDS.OnlineLDS_library.plot_p1}}\pysiglinewithargsret{\sphinxcode{\sphinxupquote{LDS.OnlineLDS\_library.}}\sphinxbfcode{\sphinxupquote{plot\_p1}}}{\emph{\DUrole{n}{ymin}}, \emph{\DUrole{n}{ymax}}, \emph{\DUrole{n}{inputs}}, \emph{\DUrole{n}{sequence\_label}}, \emph{\DUrole{n}{have\_spectral\_persistent}}, \emph{\DUrole{n}{predicted\_spectral}}, \emph{\DUrole{n}{predicted\_ar}}, \emph{\DUrole{n}{sys}}, \emph{\DUrole{n}{p\_p}}}{}
\sphinxAtStartPar
Plots seq0, seq1, seq2, logratio pdf files.
\begin{quote}\begin{description}
\item[{Parameters}] \leavevmode\begin{itemize}
\item {} 
\sphinxAtStartPar
\sphinxstyleliteralstrong{\sphinxupquote{ymin}} \textendash{} Minimal value of y\sphinxhyphen{}axis.

\item {} 
\sphinxAtStartPar
\sphinxstyleliteralstrong{\sphinxupquote{ymax}} \textendash{} Maximal value of y\sphinxhyphen{}axis.

\item {} 
\sphinxAtStartPar
\sphinxstyleliteralstrong{\sphinxupquote{inputs}} \textendash{} Input to the system matrix.

\item {} 
\sphinxAtStartPar
\sphinxstyleliteralstrong{\sphinxupquote{sequence\_label}} \textendash{} Plot’s label.

\item {} 
\sphinxAtStartPar
\sphinxstyleliteralstrong{\sphinxupquote{have\_spectral\_persistent}} \textendash{} True if we want to build spectral and persistent filters.

\item {} 
\sphinxAtStartPar
\sphinxstyleliteralstrong{\sphinxupquote{predicted\_spectral}} \textendash{} Predicted values of spectral filter. If
have\_spectral\_persistent is False, it’s equal to an empty list.

\item {} 
\sphinxAtStartPar
\sphinxstyleliteralstrong{\sphinxupquote{predicted\_ar}} \textendash{} Predicted values of auto\sphinxhyphen{}regression.

\item {} 
\sphinxAtStartPar
\sphinxstyleliteralstrong{\sphinxupquote{sys}} \textendash{} Linear Dynamical System created with DynamicalSystem class.

\item {} 
\sphinxAtStartPar
\sphinxstyleliteralstrong{\sphinxupquote{p\_p}} \textendash{} PDF file, to which are export the plots.

\end{itemize}

\end{description}\end{quote}

\end{fulllineitems}

\index{plot\_p2() (in module LDS.OnlineLDS\_library)@\spxentry{plot\_p2()}\spxextra{in module LDS.OnlineLDS\_library}}

\begin{fulllineitems}
\phantomsection\label{\detokenize{LDS:LDS.OnlineLDS_library.plot_p2}}\pysiglinewithargsret{\sphinxcode{\sphinxupquote{LDS.OnlineLDS\_library.}}\sphinxbfcode{\sphinxupquote{plot\_p2}}}{\emph{\DUrole{n}{have\_spectral\_persistent}}, \emph{\DUrole{n}{error\_spec}}, \emph{\DUrole{n}{error\_persist}}, \emph{\DUrole{n}{error\_ar}}, \emph{\DUrole{n}{lab}}, \emph{\DUrole{n}{p\_p}}}{}
\sphinxAtStartPar
Plots seq0, seq1, seq2, logratio pdf files.
\begin{quote}\begin{description}
\item[{Parameters}] \leavevmode\begin{itemize}
\item {} 
\sphinxAtStartPar
\sphinxstyleliteralstrong{\sphinxupquote{have\_spectral\_persistent}} \textendash{} True if we want to build spectral and persistent filters.

\item {} 
\sphinxAtStartPar
\sphinxstyleliteralstrong{\sphinxupquote{error\_spec}} \textendash{} Spectral filter error.

\item {} 
\sphinxAtStartPar
\sphinxstyleliteralstrong{\sphinxupquote{error\_persist}} \textendash{} Persistent filter error.

\item {} 
\sphinxAtStartPar
\sphinxstyleliteralstrong{\sphinxupquote{error\_ar}} \textendash{} Auto\sphinxhyphen{}regression error.

\item {} 
\sphinxAtStartPar
\sphinxstyleliteralstrong{\sphinxupquote{lab}} \textendash{} Auto\sphinxhyphen{}regression plot label.

\item {} 
\sphinxAtStartPar
\sphinxstyleliteralstrong{\sphinxupquote{p\_p}} \textendash{} PDF file, to which are export the plots.

\end{itemize}

\end{description}\end{quote}

\end{fulllineitems}

\index{plot\_p3() (in module LDS.OnlineLDS\_library)@\spxentry{plot\_p3()}\spxextra{in module LDS.OnlineLDS\_library}}

\begin{fulllineitems}
\phantomsection\label{\detokenize{LDS:LDS.OnlineLDS_library.plot_p3}}\pysiglinewithargsret{\sphinxcode{\sphinxupquote{LDS.OnlineLDS\_library.}}\sphinxbfcode{\sphinxupquote{plot\_p3}}}{\emph{\DUrole{n}{ymin}}, \emph{\DUrole{n}{ymax}}, \emph{\DUrole{n}{have\_spectral\_persistent}}, \emph{\DUrole{n}{error\_spec\_mean}}, \emph{\DUrole{n}{error\_spec\_std}}, \emph{\DUrole{n}{error\_persist\_mean}}, \emph{\DUrole{n}{error\_persist\_std}}, \emph{\DUrole{n}{error\_ar\_mean}}, \emph{\DUrole{n}{error\_ar\_std}}, \emph{\DUrole{n}{t\_t}}, \emph{\DUrole{n}{p\_p}}}{}
\sphinxAtStartPar
Plots seq0, seq1, seq2, logratio pdf files.
\begin{quote}\begin{description}
\item[{Parameters}] \leavevmode\begin{itemize}
\item {} 
\sphinxAtStartPar
\sphinxstyleliteralstrong{\sphinxupquote{ymin}} \textendash{} Minimal value of y\sphinxhyphen{}axis.

\item {} 
\sphinxAtStartPar
\sphinxstyleliteralstrong{\sphinxupquote{ymax}} \textendash{} Maximal value of y\sphinxhyphen{}axis.

\item {} 
\sphinxAtStartPar
\sphinxstyleliteralstrong{\sphinxupquote{have\_spectral\_persistent}} \textendash{} True if we want to build spectral and persistent filters.

\item {} 
\sphinxAtStartPar
\sphinxstyleliteralstrong{\sphinxupquote{error\_spec\_mean}} \textendash{} Mean error of spectral filtering.

\item {} 
\sphinxAtStartPar
\sphinxstyleliteralstrong{\sphinxupquote{error\_spec\_std}} \textendash{} Std of spectral filtering error.

\item {} 
\sphinxAtStartPar
\sphinxstyleliteralstrong{\sphinxupquote{error\_persist\_mean}} \textendash{} Mean error of last\sphinxhyphen{}value prediction.

\item {} 
\sphinxAtStartPar
\sphinxstyleliteralstrong{\sphinxupquote{error\_persist\_std}} \textendash{} Std of last\sphinxhyphen{}value prediction error.

\item {} 
\sphinxAtStartPar
\sphinxstyleliteralstrong{\sphinxupquote{error\_ar\_mean}} \textendash{} Mean error of auto\sphinxhyphen{}regression.

\item {} 
\sphinxAtStartPar
\sphinxstyleliteralstrong{\sphinxupquote{error\_ar\_std}} \textendash{} Std of auto\sphinxhyphen{}regression error.

\item {} 
\sphinxAtStartPar
\sphinxstyleliteralstrong{\sphinxupquote{p\_p}} \textendash{} PDF file, to which are export the plots.

\end{itemize}

\end{description}\end{quote}

\end{fulllineitems}

\index{pre\_comp\_filter\_params() (in module LDS.OnlineLDS\_library)@\spxentry{pre\_comp\_filter\_params()}\spxextra{in module LDS.OnlineLDS\_library}}

\begin{fulllineitems}
\phantomsection\label{\detokenize{LDS:LDS.OnlineLDS_library.pre_comp_filter_params}}\pysiglinewithargsret{\sphinxcode{\sphinxupquote{LDS.OnlineLDS\_library.}}\sphinxbfcode{\sphinxupquote{pre\_comp\_filter\_params}}}{\emph{\DUrole{n}{G}}, \emph{\DUrole{n}{f\_dash}}, \emph{\DUrole{n}{proc\_noise\_std}}, \emph{\DUrole{n}{obs\_noise\_std}}, \emph{\DUrole{n}{t\_t}}}{}
\sphinxAtStartPar
Kalman filter auxiliary recursive parameters calculation.

\end{fulllineitems}

\index{prediction() (in module LDS.OnlineLDS\_library)@\spxentry{prediction()}\spxextra{in module LDS.OnlineLDS\_library}}

\begin{fulllineitems}
\phantomsection\label{\detokenize{LDS:LDS.OnlineLDS_library.prediction}}\pysiglinewithargsret{\sphinxcode{\sphinxupquote{LDS.OnlineLDS\_library.}}\sphinxbfcode{\sphinxupquote{prediction}}}{\emph{\DUrole{n}{t\_t}}, \emph{\DUrole{n}{f\_dash}}, \emph{\DUrole{n}{G}}, \emph{\DUrole{n}{matrix\_a}}, \emph{\DUrole{n}{sys}}, \emph{\DUrole{n}{s}}, \emph{\DUrole{n}{Z}}, \emph{\DUrole{n}{Y}}}{}
\sphinxAtStartPar
Auto\sphinxhyphen{}regression prediction values.
Finds the formula for Figure 1(AR(s+1)):
The unrolling of the forecast \(f_{t+1}\).
The remainder term goes to zero exponentially fast with \(s\), by Lemma

\end{fulllineitems}

\index{prediction\_kalman() (in module LDS.OnlineLDS\_library)@\spxentry{prediction\_kalman()}\spxextra{in module LDS.OnlineLDS\_library}}

\begin{fulllineitems}
\phantomsection\label{\detokenize{LDS:LDS.OnlineLDS_library.prediction_kalman}}\pysiglinewithargsret{\sphinxcode{\sphinxupquote{LDS.OnlineLDS\_library.}}\sphinxbfcode{\sphinxupquote{prediction\_kalman}}}{\emph{\DUrole{n}{t\_t}}, \emph{\DUrole{n}{f\_dash}}, \emph{\DUrole{n}{G}}, \emph{\DUrole{n}{matrix\_a}}, \emph{\DUrole{n}{sys}}, \emph{\DUrole{n}{Z}}, \emph{\DUrole{n}{Y}}}{}
\sphinxAtStartPar
Kalman filter prediction values

\end{fulllineitems}

\index{testImpactOfS() (in module LDS.OnlineLDS\_library)@\spxentry{testImpactOfS()}\spxextra{in module LDS.OnlineLDS\_library}}

\begin{fulllineitems}
\phantomsection\label{\detokenize{LDS:LDS.OnlineLDS_library.testImpactOfS}}\pysiglinewithargsret{\sphinxcode{\sphinxupquote{LDS.OnlineLDS\_library.}}\sphinxbfcode{\sphinxupquote{testImpactOfS}}}{\emph{\DUrole{n}{t\_t}\DUrole{o}{=}\DUrole{default_value}{200}}, \emph{\DUrole{n}{no\_runs}\DUrole{o}{=}\DUrole{default_value}{100}}, \emph{\DUrole{n}{sMax}\DUrole{o}{=}\DUrole{default_value}{15}}}{}
\sphinxAtStartPar
Creates file ‘./outputs/impacts.pdf’, which stores plots of average error of auto\sphinxhyphen{}regression
as a function of regression depth s. In the main paper we present it again with Example 7
and \hyperref[\detokenize{LDS:fig4}]{Fig.\@ \ref{\detokenize{LDS:fig4}}} .
Increasing s decreases the error, until the error approaches that of the Kalman filter.
For a given value of the observation noise, the convergence w.r.t s is slower for
smaller process noise.
Originally the function comes from experiments.py file.

\begin{figure}[htbp]
\centering
\capstart

\noindent\sphinxincludegraphics{{Figure4}.png}
\caption{The error of AR(s + 1) as a function of s + 1, in terms of the mean
and standard deviation over N = 100 runs on Example 7, for 4 choices of \(w, v\)
of process and observation noise, respectively.}\label{\detokenize{LDS:id3}}\label{\detokenize{LDS:fig4}}\end{figure}
\begin{quote}\begin{description}
\item[{Parameters}] \leavevmode\begin{itemize}
\item {} 
\sphinxAtStartPar
\sphinxstyleliteralstrong{\sphinxupquote{t\_t}} \textendash{} Time horizon.

\item {} 
\sphinxAtStartPar
\sphinxstyleliteralstrong{\sphinxupquote{no\_runs}} \textendash{} Number of runs.

\item {} 
\sphinxAtStartPar
\sphinxstyleliteralstrong{\sphinxupquote{sMax}} \textendash{} Number of auto\sphinxhyphen{}regressive terms.

\end{itemize}

\item[{Raises}] \leavevmode
\sphinxAtStartPar
\sphinxstyleliteralstrong{\sphinxupquote{Exits if sMax \textgreater{} t\_t.}} \textendash{} 

\end{description}\end{quote}

\end{fulllineitems}

\index{testNoiseImpact() (in module LDS.OnlineLDS\_library)@\spxentry{testNoiseImpact()}\spxextra{in module LDS.OnlineLDS\_library}}

\begin{fulllineitems}
\phantomsection\label{\detokenize{LDS:LDS.OnlineLDS_library.testNoiseImpact}}\pysiglinewithargsret{\sphinxcode{\sphinxupquote{LDS.OnlineLDS\_library.}}\sphinxbfcode{\sphinxupquote{testNoiseImpact}}}{\emph{\DUrole{n}{t\_t}\DUrole{o}{=}\DUrole{default_value}{50}}, \emph{\DUrole{n}{no\_runs}\DUrole{o}{=}\DUrole{default_value}{10}}, \emph{\DUrole{n}{discretisation}\DUrole{o}{=}\DUrole{default_value}{10}}}{}
\sphinxAtStartPar
Produces ‘./outputs/noise.pdf’. Plots heatmap of process noise variance
vs observation noise variance based on relative error between any two
predictive algorithms. LaTeX shows the example of the ratio of the errors
of Kalman filter and AR(2)(see \hyperref[\detokenize{LDS:fig3}]{Fig.\@ \ref{\detokenize{LDS:fig3}}}).
Originally the function comes from experiments.py file.

\sphinxAtStartPar
Plots RMSE of AR \hyperref[\detokenize{LDS:fig6}]{Fig.\@ \ref{\detokenize{LDS:fig6}}} (left):
average RMSE of predictions of AR(s+ 1) as a function of the variance of the
process noise (vertical axis) and observation noise (horizontal axis).

\sphinxAtStartPar
Plots \hyperref[\detokenize{LDS:fig6}]{Fig.\@ \ref{\detokenize{LDS:fig6}}} (center):
The differences in average RMSE of Kalman filters and AR(s + 1) as a function
of the variance of the process noise (vertical axis) and observation noise (horizontal axis).

\sphinxAtStartPar
Plots \hyperref[\detokenize{LDS:fig6}]{Fig.\@ \ref{\detokenize{LDS:fig6}}} (right):
The ratio (70) of the errors of Kalman filters and AR(s + 1) as a function of
the variance of the process noise (vertical axis) and observation noise (horizontal axis).
\begin{quote}\begin{description}
\item[{Parameters}] \leavevmode\begin{itemize}
\item {} 
\sphinxAtStartPar
\sphinxstyleliteralstrong{\sphinxupquote{t\_t}} \textendash{} Time horizon.

\item {} 
\sphinxAtStartPar
\sphinxstyleliteralstrong{\sphinxupquote{no\_runs}} \textendash{} Number of runs.

\item {} 
\sphinxAtStartPar
\sphinxstyleliteralstrong{\sphinxupquote{discretisation}} \textendash{} Number of trajectories.

\end{itemize}

\end{description}\end{quote}

\end{fulllineitems}

\index{testSeqD0() (in module LDS.OnlineLDS\_library)@\spxentry{testSeqD0()}\spxextra{in module LDS.OnlineLDS\_library}}

\begin{fulllineitems}
\phantomsection\label{\detokenize{LDS:LDS.OnlineLDS_library.testSeqD0}}\pysiglinewithargsret{\sphinxcode{\sphinxupquote{LDS.OnlineLDS\_library.}}\sphinxbfcode{\sphinxupquote{testSeqD0}}}{\emph{\DUrole{n}{no\_runs}\DUrole{o}{=}\DUrole{default_value}{100}}}{}
\sphinxAtStartPar
Makes several initiations of test\_identification function so as to plot “logratio.pdf” and
“seq0.pdf”, “seq1.pdf”, “seq2.pdf”. Originally the function comes from experiments.py file.
\begin{quote}\begin{description}
\item[{Parameters}] \leavevmode
\sphinxAtStartPar
\sphinxstyleliteralstrong{\sphinxupquote{no\_runs}} \textendash{} Number of runs.

\end{description}\end{quote}

\end{fulllineitems}

\index{test\_AR() (in module LDS.OnlineLDS\_library)@\spxentry{test\_AR()}\spxextra{in module LDS.OnlineLDS\_library}}

\begin{fulllineitems}
\phantomsection\label{\detokenize{LDS:LDS.OnlineLDS_library.test_AR}}\pysiglinewithargsret{\sphinxcode{\sphinxupquote{LDS.OnlineLDS\_library.}}\sphinxbfcode{\sphinxupquote{test\_AR}}}{}{}
\sphinxAtStartPar
Function implements Algorithm 1(On\sphinxhyphen{}line Gradient Descent).
Originally the function comes from experiments.py file.

\end{fulllineitems}

\index{test\_arima\_ogd() (in module LDS.OnlineLDS\_library)@\spxentry{test\_arima\_ogd()}\spxextra{in module LDS.OnlineLDS\_library}}

\begin{fulllineitems}
\phantomsection\label{\detokenize{LDS:LDS.OnlineLDS_library.test_arima_ogd}}\pysiglinewithargsret{\sphinxcode{\sphinxupquote{LDS.OnlineLDS\_library.}}\sphinxbfcode{\sphinxupquote{test\_arima\_ogd}}}{\emph{\DUrole{n}{i}}, \emph{\DUrole{n}{mk}}, \emph{\DUrole{n}{lrate}}, \emph{\DUrole{n}{data}}}{}
\sphinxAtStartPar
Used to test arima\_ogd function for i=10 case.
The test cases are based on MATLAB:
The test numbers were taken from the output of MATLAB function,
the random array w is fixed.
\begin{quote}\begin{description}
\item[{Parameters}] \leavevmode\begin{itemize}
\item {} 
\sphinxAtStartPar
\sphinxstyleliteralstrong{\sphinxupquote{i}} \textendash{} Iterative number. In range from mk till data \sphinxhyphen{} 1.

\item {} 
\sphinxAtStartPar
\sphinxstyleliteralstrong{\sphinxupquote{mk}} \textendash{} Integer number. Here we used 10.

\item {} 
\sphinxAtStartPar
\sphinxstyleliteralstrong{\sphinxupquote{lrate}} \textendash{} Learning rate. Assigned 1 in example.py.

\item {} 
\sphinxAtStartPar
\sphinxstyleliteralstrong{\sphinxupquote{data}} \textendash{} Array of 10000 elements.

\end{itemize}

\end{description}\end{quote}

\sphinxAtStartPar
Raises:

\end{fulllineitems}

\index{test\_arima\_ons() (in module LDS.OnlineLDS\_library)@\spxentry{test\_arima\_ons()}\spxextra{in module LDS.OnlineLDS\_library}}

\begin{fulllineitems}
\phantomsection\label{\detokenize{LDS:LDS.OnlineLDS_library.test_arima_ons}}\pysiglinewithargsret{\sphinxcode{\sphinxupquote{LDS.OnlineLDS\_library.}}\sphinxbfcode{\sphinxupquote{test\_arima\_ons}}}{\emph{\DUrole{n}{i}}, \emph{\DUrole{n}{mk}}, \emph{\DUrole{n}{lrate}}, \emph{\DUrole{n}{data}}, \emph{\DUrole{n}{A\_trans\_in}}}{}
\sphinxAtStartPar
to test arima\_ons function
the test casees are based on MATLAB:
the test numbers were taken from the output of MATLAB function
the random array w is fixed
\begin{quote}\begin{description}
\item[{Parameters}] \leavevmode\begin{itemize}
\item {} 
\sphinxAtStartPar
\sphinxstyleliteralstrong{\sphinxupquote{i}} \textendash{} 

\item {} 
\sphinxAtStartPar
\sphinxstyleliteralstrong{\sphinxupquote{mk}} \textendash{} 

\item {} 
\sphinxAtStartPar
\sphinxstyleliteralstrong{\sphinxupquote{lrate}} \textendash{} 

\item {} 
\sphinxAtStartPar
\sphinxstyleliteralstrong{\sphinxupquote{data}} \textendash{} 

\end{itemize}

\item[{Returns}] \leavevmode
\sphinxAtStartPar


\end{description}\end{quote}

\end{fulllineitems}

\index{test\_identification() (in module LDS.OnlineLDS\_library)@\spxentry{test\_identification()}\spxextra{in module LDS.OnlineLDS\_library}}

\begin{fulllineitems}
\phantomsection\label{\detokenize{LDS:LDS.OnlineLDS_library.test_identification}}\pysiglinewithargsret{\sphinxcode{\sphinxupquote{LDS.OnlineLDS\_library.}}\sphinxbfcode{\sphinxupquote{test\_identification}}}{\emph{\DUrole{n}{sys}}, \emph{\DUrole{n}{filename\_stub}\DUrole{o}{=}\DUrole{default_value}{\textquotesingle{}test\textquotesingle{}}}, \emph{\DUrole{n}{no\_runs}\DUrole{o}{=}\DUrole{default_value}{2}}, \emph{\DUrole{n}{t\_t}\DUrole{o}{=}\DUrole{default_value}{100}}, \emph{\DUrole{n}{k}\DUrole{o}{=}\DUrole{default_value}{5}}, \emph{\DUrole{n}{eta\_zeros}\DUrole{o}{=}\DUrole{default_value}{None}}, \emph{\DUrole{n}{ymin}\DUrole{o}{=}\DUrole{default_value}{None}}, \emph{\DUrole{n}{ymax}\DUrole{o}{=}\DUrole{default_value}{None}}, \emph{\DUrole{n}{sequence\_label}\DUrole{o}{=}\DUrole{default_value}{None}}, \emph{\DUrole{n}{have\_spectral\_persistent}\DUrole{o}{=}\DUrole{default_value}{True}}}{}
\sphinxAtStartPar
Implements here On\sphinxhyphen{}line Gradient Descent Algorithm 1 by the use of cost\_ar and gradient\_ar
functions.
Data found is used by plot\_p1,plot\_p2, plot\_p3 functions which create “seq0”, “logration” pdfs.
Implements Example 8 from Experiments section of Marecek’s paper.
Originally the function comes from experiments.py file. Plots \hyperref[\detokenize{LDS:fig7}]{Fig.\@ \ref{\detokenize{LDS:fig7}}}, \hyperref[\detokenize{LDS:fig8}]{Fig.\@ \ref{\detokenize{LDS:fig8}}}.

\begin{figure}[htbp]
\centering
\capstart

\noindent\sphinxincludegraphics{{Figure7}.png}
\caption{An illustration of the impact of constants in the learning rate on
Example 8 from Marecek’s paper, the well\sphinxhyphen{}known time series.
Top: The forecasts for three different values of c.
Bottom: The error for three different values of c.}\label{\detokenize{LDS:id4}}\label{\detokenize{LDS:fig7}}\end{figure}

\begin{figure}[htbp]
\centering
\capstart

\noindent\sphinxincludegraphics{{Figure8}.png}
\caption{Illustrations on Example 8, the well\sphinxhyphen{}known time series. Top: the
predictions of AR(2) compared with the predictions of the spectral filter of
Hazan, Singh, and Zhang (2017) and the trivial last\sphinxhyphen{}value prediction on the
first T = 100 elements of series d0 (left), d1 (center), and d2 (right).
Bottom: the corresponding errors.}\label{\detokenize{LDS:id5}}\label{\detokenize{LDS:fig8}}\end{figure}
\begin{quote}\begin{description}
\item[{Parameters}] \leavevmode\begin{itemize}
\item {} 
\sphinxAtStartPar
\sphinxstyleliteralstrong{\sphinxupquote{sys}} \textendash{} LDS.

\item {} 
\sphinxAtStartPar
\sphinxstyleliteralstrong{\sphinxupquote{filename\_stub}} \textendash{} Name of the output file.

\item {} 
\sphinxAtStartPar
\sphinxstyleliteralstrong{\sphinxupquote{no\_runs}} \textendash{} Number of runs.

\item {} 
\sphinxAtStartPar
\sphinxstyleliteralstrong{\sphinxupquote{t\_t}} \textendash{} Time horizon.

\item {} 
\sphinxAtStartPar
\sphinxstyleliteralstrong{\sphinxupquote{k}} \textendash{} Number of filters.

\item {} 
\sphinxAtStartPar
\sphinxstyleliteralstrong{\sphinxupquote{eta\_zeros}} \textendash{} Learning rate.

\item {} 
\sphinxAtStartPar
\sphinxstyleliteralstrong{\sphinxupquote{y\_min}} \textendash{} Minimal value of y\sphinxhyphen{}axis.

\item {} 
\sphinxAtStartPar
\sphinxstyleliteralstrong{\sphinxupquote{y\_max}} \textendash{} Maximal value of y\sphinxhyphen{}axis.

\item {} 
\sphinxAtStartPar
\sphinxstyleliteralstrong{\sphinxupquote{sequence\_label}} \textendash{} 

\item {} 
\sphinxAtStartPar
\sphinxstyleliteralstrong{\sphinxupquote{have\_spectral\_persistent}} \textendash{} False if there’s no need to plot spectral and persistent filters.
Default value \sphinxhyphen{} True.

\end{itemize}

\item[{Raises}] \leavevmode
\sphinxAtStartPar
\sphinxstyleliteralstrong{\sphinxupquote{Exits if k \textgreater{} t\_t.}} \textendash{} 

\end{description}\end{quote}

\end{fulllineitems}

\index{test\_identification2() (in module LDS.OnlineLDS\_library)@\spxentry{test\_identification2()}\spxextra{in module LDS.OnlineLDS\_library}}

\begin{fulllineitems}
\phantomsection\label{\detokenize{LDS:LDS.OnlineLDS_library.test_identification2}}\pysiglinewithargsret{\sphinxcode{\sphinxupquote{LDS.OnlineLDS\_library.}}\sphinxbfcode{\sphinxupquote{test\_identification2}}}{\emph{\DUrole{n}{t\_t}\DUrole{o}{=}\DUrole{default_value}{100}}, \emph{\DUrole{n}{no\_runs}\DUrole{o}{=}\DUrole{default_value}{10}}, \emph{\DUrole{n}{s\_choices}\DUrole{o}{=}\DUrole{default_value}{{[}15, 3, 1{]}}}, \emph{\DUrole{n}{have\_kalman}\DUrole{o}{=}\DUrole{default_value}{False}}, \emph{\DUrole{n}{have\_spectral\_persistent}\DUrole{o}{=}\DUrole{default_value}{True}}, \emph{\DUrole{n}{G}\DUrole{o}{=}\DUrole{default_value}{array({[}{[}0.47818304, 0.63191781{]}, {[}0.71975662, 0.51588563{]}{]})}}, \emph{\DUrole{n}{f\_dash}\DUrole{o}{=}\DUrole{default_value}{array({[}{[}0.77427218, 0.8161933{]}{]})}}, \emph{\DUrole{n}{sequence\_label}\DUrole{o}{=}\DUrole{default_value}{\textquotesingle{}\textquotesingle{}}}}{}
\sphinxAtStartPar
Implements Example 7 from Experiments section of the paper.
Creates ‘./outputs/AR.pdf’.Finds all the filters’ errors and
uses function p3\_for\_test\_identification2 for plotting them.
Plots \hyperref[\detokenize{LDS:fig2}]{Fig.\@ \ref{\detokenize{LDS:fig2}}}, \hyperref[\detokenize{LDS:fig5}]{Fig.\@ \ref{\detokenize{LDS:fig5}}}. Plots Figure 2,5 of the main paper.
Originally the function comes from experiments.py file.

\begin{figure}[htbp]
\centering
\capstart

\noindent\sphinxincludegraphics{{Figure2}.png}
\caption{The error of AR(2) compared against Kalman filter, last\sphinxhyphen{}value prediction,
and spectral filtering in terms of the mean and standard deviation over
N = 100 runs on Example 7.}\label{\detokenize{LDS:id6}}\label{\detokenize{LDS:fig2}}\end{figure}

\begin{figure}[htbp]
\centering
\capstart

\noindent\sphinxincludegraphics{{Figure5}.png}
\caption{Left: sample outputs and predictions with AR(2), compared against Kalman filter,
last\sphinxhyphen{}value prediction, and spectral filtering of Hazan, Singh, and Zhang (2017).
Right: Same as \hyperref[\detokenize{LDS:fig2}]{Fig.\@ \ref{\detokenize{LDS:fig2}}}, over longer time period.}\label{\detokenize{LDS:id7}}\label{\detokenize{LDS:fig5}}\end{figure}
\begin{quote}\begin{description}
\item[{Parameters}] \leavevmode\begin{itemize}
\item {} 
\sphinxAtStartPar
\sphinxstyleliteralstrong{\sphinxupquote{t\_t}} \textendash{} Time horizon.

\item {} 
\sphinxAtStartPar
\sphinxstyleliteralstrong{\sphinxupquote{no\_runs}} \textendash{} Number of runs.

\item {} 
\sphinxAtStartPar
\sphinxstyleliteralstrong{\sphinxupquote{s\_choices}} \textendash{} 

\item {} 
\sphinxAtStartPar
\sphinxstyleliteralstrong{\sphinxupquote{have\_kalman}} \textendash{} False if there’s no need to plot kalman filter.
Default value \sphinxhyphen{} True.

\item {} 
\sphinxAtStartPar
\sphinxstyleliteralstrong{\sphinxupquote{have\_spectral\_persistent}} \textendash{} False if there’s no need to plot spectral and persistent filters.
Default value \sphinxhyphen{} True.

\item {} 
\sphinxAtStartPar
\sphinxstyleliteralstrong{\sphinxupquote{G}} \textendash{} State matrix.

\item {} 
\sphinxAtStartPar
\sphinxstyleliteralstrong{\sphinxupquote{f\_dash}} \textendash{} first derivative of the observation direction.

\item {} 
\sphinxAtStartPar
\sphinxstyleliteralstrong{\sphinxupquote{sequence\_label}} \textendash{} 

\end{itemize}

\item[{Raises}] \leavevmode
\sphinxAtStartPar
\sphinxstyleliteralstrong{\sphinxupquote{Exits if number of runs is less than 2.}} \textendash{} 

\end{description}\end{quote}

\end{fulllineitems}

\index{w\_calc() (in module LDS.OnlineLDS\_library)@\spxentry{w\_calc()}\spxextra{in module LDS.OnlineLDS\_library}}

\begin{fulllineitems}
\phantomsection\label{\detokenize{LDS:LDS.OnlineLDS_library.w_calc}}\pysiglinewithargsret{\sphinxcode{\sphinxupquote{LDS.OnlineLDS\_library.}}\sphinxbfcode{\sphinxupquote{w\_calc}}}{\emph{\DUrole{n}{w}}, \emph{\DUrole{n}{data}}, \emph{\DUrole{n}{mk}}, \emph{\DUrole{n}{i}}, \emph{\DUrole{n}{diff}}, \emph{\DUrole{n}{lrate}}}{}
\sphinxAtStartPar
Auxiliary function to implement ARIMA in python. Others functions use it in their
iterations.
MATLAB: w = w \sphinxhyphen{} data(i\sphinxhyphen{}mk:i\sphinxhyphen{}1)*2*diff/sqrt(i\sphinxhyphen{}mk)*lrate;
\begin{quote}\begin{description}
\item[{Parameters}] \leavevmode\begin{itemize}
\item {} 
\sphinxAtStartPar
\sphinxstyleliteralstrong{\sphinxupquote{w}} \textendash{} Uniform distribution array with options.mk number of columns.

\item {} 
\sphinxAtStartPar
\sphinxstyleliteralstrong{\sphinxupquote{data}} \textendash{} Array of 10000 elements.

\item {} 
\sphinxAtStartPar
\sphinxstyleliteralstrong{\sphinxupquote{mk}} \textendash{} Integer number. Here we used 10.

\item {} 
\sphinxAtStartPar
\sphinxstyleliteralstrong{\sphinxupquote{i}} \textendash{} Iterative number. In range from mk till data \sphinxhyphen{} 1.

\item {} 
\sphinxAtStartPar
\sphinxstyleliteralstrong{\sphinxupquote{diff}} \textendash{} Result of diff\_calc function

\item {} 
\sphinxAtStartPar
\sphinxstyleliteralstrong{\sphinxupquote{lrate}} \textendash{} Learning rate. Assigned 1 in example.py.

\end{itemize}

\end{description}\end{quote}

\sphinxAtStartPar
Returns:

\end{fulllineitems}

\index{w\_calc\_arima\_ons() (in module LDS.OnlineLDS\_library)@\spxentry{w\_calc\_arima\_ons()}\spxextra{in module LDS.OnlineLDS\_library}}

\begin{fulllineitems}
\phantomsection\label{\detokenize{LDS:LDS.OnlineLDS_library.w_calc_arima_ons}}\pysiglinewithargsret{\sphinxcode{\sphinxupquote{LDS.OnlineLDS\_library.}}\sphinxbfcode{\sphinxupquote{w\_calc\_arima\_ons}}}{\emph{\DUrole{n}{w}}, \emph{\DUrole{n}{lrate}}, \emph{\DUrole{n}{grad}}, \emph{\DUrole{n}{A\_trans}}}{}
\sphinxAtStartPar
MATLAB: w = w \sphinxhyphen{} lrate * grad * A\_trans
Calculation of the weight with Gradient Descent algorithm.
\begin{quote}\begin{description}
\item[{Parameters}] \leavevmode\begin{itemize}
\item {} 
\sphinxAtStartPar
\sphinxstyleliteralstrong{\sphinxupquote{w}} \textendash{} Uniform distribution array with options.mk number of columns.

\item {} 
\sphinxAtStartPar
\sphinxstyleliteralstrong{\sphinxupquote{lrate}} \textendash{} Learning rate. Assigned 1 in example.py.

\item {} 
\sphinxAtStartPar
\sphinxstyleliteralstrong{\sphinxupquote{grad}} \textendash{} Gradient, the return of the function grad\_calc.

\item {} 
\sphinxAtStartPar
\sphinxstyleliteralstrong{\sphinxupquote{A\_trans}} \textendash{} Return of the function A\_trans\_calc.

\end{itemize}

\item[{Returns}] \leavevmode
\sphinxAtStartPar
Weight after an iteration of the gradient descent algorithm.

\end{description}\end{quote}

\end{fulllineitems}



\subsection{LDS.example module}
\label{\detokenize{LDS:lds-example-module}}

\subsection{LDS.main module}
\label{\detokenize{LDS:module-LDS.main}}\label{\detokenize{LDS:lds-main-module}}\index{module@\spxentry{module}!LDS.main@\spxentry{LDS.main}}\index{LDS.main@\spxentry{LDS.main}!module@\spxentry{module}}

\subsection{Module contents}
\label{\detokenize{LDS:module-LDS}}\label{\detokenize{LDS:module-contents}}\index{module@\spxentry{module}!LDS@\spxentry{LDS}}\index{LDS@\spxentry{LDS}!module@\spxentry{module}}

\chapter{Indices and tables}
\label{\detokenize{index:indices-and-tables}}\begin{itemize}
\item {} 
\sphinxAtStartPar
\DUrole{xref,std,std-ref}{genindex}

\item {} 
\sphinxAtStartPar
\DUrole{xref,std,std-ref}{modindex}

\item {} 
\sphinxAtStartPar
\DUrole{xref,std,std-ref}{search}

\end{itemize}


\renewcommand{\indexname}{Python Module Index}
\begin{sphinxtheindex}
\let\bigletter\sphinxstyleindexlettergroup
\bigletter{l}
\item\relax\sphinxstyleindexentry{LDS}\sphinxstyleindexpageref{LDS:\detokenize{module-LDS}}
\item\relax\sphinxstyleindexentry{LDS.LDS}\sphinxstyleindexpageref{LDS.LDS:\detokenize{module-LDS.LDS}}
\item\relax\sphinxstyleindexentry{LDS.LDS.ds}\sphinxstyleindexpageref{LDS.LDS.ds:\detokenize{module-LDS.LDS.ds}}
\item\relax\sphinxstyleindexentry{LDS.LDS.ds.dynamical\_system}\sphinxstyleindexpageref{LDS.LDS.ds:\detokenize{module-LDS.LDS.ds.dynamical_system}}
\item\relax\sphinxstyleindexentry{LDS.LDS.filters}\sphinxstyleindexpageref{LDS.LDS.filters:\detokenize{module-LDS.LDS.filters}}
\item\relax\sphinxstyleindexentry{LDS.LDS.filters.filtering\_abc\_class}\sphinxstyleindexpageref{LDS.LDS.filters:\detokenize{module-LDS.LDS.filters.filtering_abc_class}}
\item\relax\sphinxstyleindexentry{LDS.LDS.filters.filtering\_siso}\sphinxstyleindexpageref{LDS.LDS.filters:\detokenize{module-LDS.LDS.filters.filtering_siso}}
\item\relax\sphinxstyleindexentry{LDS.LDS.filters.kalman\_em}\sphinxstyleindexpageref{LDS.LDS.filters:\detokenize{module-LDS.LDS.filters.kalman_em}}
\item\relax\sphinxstyleindexentry{LDS.LDS.filters.kalman\_filtering\_siso}\sphinxstyleindexpageref{LDS.LDS.filters:\detokenize{module-LDS.LDS.filters.kalman_filtering_siso}}
\item\relax\sphinxstyleindexentry{LDS.LDS.filters.wave\_filtering\_siso}\sphinxstyleindexpageref{LDS.LDS.filters:\detokenize{module-LDS.LDS.filters.wave_filtering_siso}}
\item\relax\sphinxstyleindexentry{LDS.LDS.filters.wave\_filtering\_siso\_abs}\sphinxstyleindexpageref{LDS.LDS.filters:\detokenize{module-LDS.LDS.filters.wave_filtering_siso_abs}}
\item\relax\sphinxstyleindexentry{LDS.LDS.filters.wave\_filtering\_siso\_ftl}\sphinxstyleindexpageref{LDS.LDS.filters:\detokenize{module-LDS.LDS.filters.wave_filtering_siso_ftl}}
\item\relax\sphinxstyleindexentry{LDS.LDS.filters.wave\_filtering\_siso\_ftl\_persistent}\sphinxstyleindexpageref{LDS.LDS.filters:\detokenize{module-LDS.LDS.filters.wave_filtering_siso_ftl_persistent}}
\item\relax\sphinxstyleindexentry{LDS.LDS.filters.wave\_filtering\_siso\_persistent}\sphinxstyleindexpageref{LDS.LDS.filters:\detokenize{module-LDS.LDS.filters.wave_filtering_siso_persistent}}
\item\relax\sphinxstyleindexentry{LDS.LDS.h\_m}\sphinxstyleindexpageref{LDS.LDS.h_m:\detokenize{module-LDS.LDS.h_m}}
\item\relax\sphinxstyleindexentry{LDS.LDS.h\_m.hankel}\sphinxstyleindexpageref{LDS.LDS.h_m:\detokenize{module-LDS.LDS.h_m.hankel}}
\item\relax\sphinxstyleindexentry{LDS.LDS.matlab\_options}\sphinxstyleindexpageref{LDS.LDS.matlab_options:\detokenize{module-LDS.LDS.matlab_options}}
\item\relax\sphinxstyleindexentry{LDS.LDS.matlab\_options.matlab\_class\_options}\sphinxstyleindexpageref{LDS.LDS.matlab_options:\detokenize{module-LDS.LDS.matlab_options.matlab_class_options}}
\item\relax\sphinxstyleindexentry{LDS.LDS.online\_lds}\sphinxstyleindexpageref{LDS.LDS.online_lds:\detokenize{module-LDS.LDS.online_lds}}
\item\relax\sphinxstyleindexentry{LDS.LDS.online\_lds.cost\_ftl}\sphinxstyleindexpageref{LDS.LDS.online_lds:\detokenize{module-LDS.LDS.online_lds.cost_ftl}}
\item\relax\sphinxstyleindexentry{LDS.LDS.online\_lds.gradient\_ftl}\sphinxstyleindexpageref{LDS.LDS.online_lds:\detokenize{module-LDS.LDS.online_lds.gradient_ftl}}
\item\relax\sphinxstyleindexentry{LDS.LDS.online\_lds.print\_verbose}\sphinxstyleindexpageref{LDS.LDS.online_lds:\detokenize{module-LDS.LDS.online_lds.print_verbose}}
\item\relax\sphinxstyleindexentry{LDS.LDS.ts}\sphinxstyleindexpageref{LDS.LDS.ts:\detokenize{module-LDS.LDS.ts}}
\item\relax\sphinxstyleindexentry{LDS.LDS.ts.time\_series}\sphinxstyleindexpageref{LDS.LDS.ts:\detokenize{module-LDS.LDS.ts.time_series}}
\item\relax\sphinxstyleindexentry{LDS.main}\sphinxstyleindexpageref{LDS:\detokenize{module-LDS.main}}
\item\relax\sphinxstyleindexentry{LDS.OnlineLDS\_library}\sphinxstyleindexpageref{LDS:\detokenize{module-LDS.OnlineLDS_library}}
\end{sphinxtheindex}

\renewcommand{\indexname}{Index}
\printindex
\end{document}